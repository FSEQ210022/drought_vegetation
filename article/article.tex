\documentclass[preprint,
3p]{elsarticle} %review=doublespace preprint=single 5p=2 column
%%% Begin My package additions %%%%%%%%%%%%%%%%%%%

\usepackage[hyphens]{url}

  \journal{An awesome journal} % Sets Journal name

\usepackage{graphicx}
%%%%%%%%%%%%%%%% end my additions to header

\usepackage[T1]{fontenc}
\usepackage{lmodern}
\usepackage{amssymb,amsmath}
% TODO: Currently lineno needs to be loaded after amsmath because of conflict
% https://github.com/latex-lineno/lineno/issues/5
\usepackage{lineno} % add
\usepackage{ifxetex,ifluatex}
\usepackage{fixltx2e} % provides \textsubscript
% use upquote if available, for straight quotes in verbatim environments
\IfFileExists{upquote.sty}{\usepackage{upquote}}{}
\ifnum 0\ifxetex 1\fi\ifluatex 1\fi=0 % if pdftex
  \usepackage[utf8]{inputenc}
\else % if luatex or xelatex
  \usepackage{fontspec}
  \ifxetex
    \usepackage{xltxtra,xunicode}
  \fi
  \defaultfontfeatures{Mapping=tex-text,Scale=MatchLowercase}
  \newcommand{\euro}{€}
\fi
% use microtype if available
\IfFileExists{microtype.sty}{\usepackage{microtype}}{}
\usepackage[]{natbib}
\bibliographystyle{apalike}

\ifxetex
  \usepackage[setpagesize=false, % page size defined by xetex
              unicode=false, % unicode breaks when used with xetex
              xetex]{hyperref}
\else
  \usepackage[unicode=true]{hyperref}
\fi
\hypersetup{breaklinks=true,
            bookmarks=true,
            pdfauthor={},
            pdftitle={Comprehensive assessment of the climate-induced water scarcity over continental Chile},
            colorlinks=false,
            urlcolor=blue,
            linkcolor=magenta,
            pdfborder={0 0 0}}

\setcounter{secnumdepth}{5}
% Pandoc toggle for numbering sections (defaults to be off)


% tightlist command for lists without linebreak
\providecommand{\tightlist}{%
  \setlength{\itemsep}{0pt}\setlength{\parskip}{0pt}}




\usepackage{booktabs}
\usepackage{longtable}
\usepackage{array}
\usepackage{multirow}
\usepackage{wrapfig}
\usepackage{float}
\usepackage{colortbl}
\usepackage{pdflscape}
\usepackage{tabu}
\usepackage{threeparttable}
\usepackage{threeparttablex}
\usepackage[normalem]{ulem}
\usepackage{makecell}
\usepackage{xcolor}



\begin{document}


\begin{frontmatter}

  \title{Comprehensive assessment of the climate-induced water scarcity
over continental Chile}
    \author[Hemere Centro de Observación de la Tierra,Universidad
Mayor]{Francisco Zambrano%
  \corref{cor1}%
  \fnref{1}}
   \ead{francisco.zambrano@umayor.com} 
    \author[Another University]{Bob Security%
  %
  }
   \ead{bob@example.com} 
    \author[Another University]{Cat Memes%
  %
  \fnref{2}}
   \ead{cat@example.com} 
    \author[Some Institute of Technology]{Derek Zoolander%
  %
  \fnref{2}}
   \ead{derek@example.com} 
      \affiliation[Some Institute of Technology]{
    organization={Big Wig University},addressline={1 main
street},city={Gotham},postcode={123456},state={State},country={United
States},}
    \affiliation[Another University]{
    organization={Department},addressline={A street
29},city={Manchester,},postcode={2054 NX},country={The Netherlands},}
    \cortext[cor1]{Corresponding author}
    \fntext[1]{This is the first author footnote.}
    \fntext[2]{Another author footnote.}
  
  \begin{abstract}
  This is the abstract.

  It consists of two paragraphs.
  \end{abstract}
    \begin{keyword}
    keyword1 \sep 
    keyword2
  \end{keyword}
  
 \end{frontmatter}

\hypertarget{version}{%
\section{Version}\label{version}}

This Rmd-skeleton uses the mdpi Latex template published 2019/02.
However, the official template gets more frequently updated than the
`rticles' package. Therefore, please make sure prior to paper
submission, that you're using the most recent .cls, .tex and .bst files
(available \href{http://www.mdpi.com/authors/latex}{here}).

\hypertarget{introduction}{%
\section{Introduction}\label{introduction}}

In 2021, the sixth assessment report (AR6) from the working group I of
the IPCC was released \citep{IPCC2021}. Chapter 11 \citep{IPCCCH112021}
indicates that human-induced greenhouse gas emissions have increased the
frequency and/or intensity of some weather and climate extremes. The
evidence has been strengthened since AR5 \citep{IPCC2013}. There is high
confidence that the increasing global warning can expand the land area
affected by increasing drought frequency and severity
\citep{IPCCCH112021}. Chile has been facing a persistent rainfall
deficit lasting for more than ten years \citep{Garreaud2017} which has
impacted the hydrological system \citep{Boisier2018}, and consequently
the vegetation development \citep{Zambrano2020}.

Precipitation is the primary driver of drought that impacts hydrological
regimes and vegetation productivity. Thus, it is commonly classified as
meteorological, hydrological, and agricultural \citep{Wilhite1985}.
Lately, it has been argued that this definition does not fully address
the ecological dimensions \citep{Crausbay2017}. \citet{Crausbay2017}
proposed the ecological drought definition as ``an episodic deficit in
water availability that drives ecosystems beyond thresholds of
vulnerability, impacts ecosystem services, and triggers feedback in
natural and/or human systems''. The AR6 \citep{IPCC2021} state that even
if global warming is stabilized at 1.5°-2°C many parts of the world will
be impacted by more severe agricultural and ecological drought. Central
Chile has suffered from crop productivity failure, highlighting the
growing season 2007-2008 and 2008-2009
\citep{Zambrano2016, Zambrano2018}, which impacted an extensive surface.
But, in 2019-2020, the drought intensity reached an extreme condition at
North 34°S not seen -at least- for more than 40 years
\citep{Zambrano2020}, affecting forest, grassland, and croplands areas.
The prolonged lack of precipitation within Central Chile is producing
changes in the ecosystem that should study.

Satellite remote sensing \citep{West2019, AghaKouchak2015} is the
primary method to evaluate how meteorological drought impacts vegetation
dynamics. Since the 90's multiple vegetation drought indices have been
derived (VCI,\citep{Kogan1990}; TCI, \citep{Kogan1995};zNDVI,
\citep{Peters2002}; VegDri, \citep{Brown2008}) that have allowed making
spatiotemporal analysis. Although we can calculate those indices for any
time during the year (depending on satellite revisit), there are
relevant during the stage vegetation has more activity, the growing
season \citep{Mishra2015}. Although modeling phenology is a complex
task, satellites offer strategies that help to address it
\citep{Younes2021, Vrieling2018, Cai2017}. Also, the land cover dynamics
product MCD12Q2 from the USGS \citep{Friedl2019} provides some phenology
metrics. Some authors have proposed indices aggregated during the
season. \citet{Meroni2017} accumulating the fractional active
photosynthetic active radiation(FAPAR) between the start (SOS) and the
end of the season (EOS) in the Sahel, calculate the zCFAPAR.
\citet{Zambrano2018} used the same approach but with the NDVI
(Normalized Difference Vegetation Index), derivating the zcNDVI within
Central Chile. Besides, land use land cover (LULC) change can be driven
by drought \citep{Tran2019, Akinyemi2021}. To analyze those changes,
multiple time-series LULC products exist as the MCD12Q1
\citep{Friedl2019} and the ESA CCI-LC \citep{ESA2017}. For Chile, 2014
was made a high-resolution land cover at 30m of spatial resolution
\citep{Zhao2016}. The LULC product with the vegetation drought index can
help evaluate the impact of drought on the ecosystem.

Vegetation drought indices are proxies of productivity
\citep{Paruelo2016, Schucknecht2017}. The main environmental variables
that affect it are water supply and demand \citep{Mishra2015}. We
measure them by precipitation and evapotranspiration (ET), commonly
collected from weather stations. Usually, in developing countries (i.e.,
Chile), incomplete records or gaps present a challenge. But, there are
satellite estimates of these variables. To evaluate drought, the World
Meteorological Organization (WMO; \citep{WMO2012}) has proposed the
Standardized Precipitation Index (SPI; \citep{McKee1993}), a multiscalar
drought index, which has been used worldwide. For Chile,
\citet{Zambrano2017} derived and evaluated it from the product of the
Climate Hazards Group InfraRed Precipitation with Station data (CHIRPS;
\citep{Funk2015}). For water demand, it is used ET. The vegetation
biomass productivity is strongly related to ET \citep{FAO66}. The
atmospheric evaporative demand (AED) represents the maximum ET rate from
a land surface (without water restriction), also known as reference ET.
The recommended method for its calculation is the FAO Penman-Monteith
\citep{Pereira2015, Allen2005}. Due to climate change, AED is
increasing, driving ET rise \citep{IPCCCH112021}. But, it is not always
true \citep{Milly2016}. For example, regions where AET is highest have
the lowest ET. The MOD16 product \citep{Running2021, Mu2011} provides
AET and ET satellite estimates and has been used to derive drought
indices \citep{Mu2013}. Soil moisture (SM) is an Essential Climate
Variable (ECV) that modulates vegetative growth. The climate change
initiative (CCI) from the European Space Agency (ESA) delivers the ESA
CCI SM product \citep{Dorigo2017} (current version 6.1), which has been
helpful to monitor drought \citep{Zhang2019}. Besides, total water
storage can be retrieved by the Gravity Recovery and Climate Experiment
(GRACE), which allows analyzing water availability changes
\citep{Ahmed2014, Ma2017}. The water demand and supply by remote sensing
can help evaluate how they have impacted vegetation productivity.

The study aims to analyze the drought impact on vegetation through
Central Chile for 2000-2020, using satellite data as proxies of
productivity and water demand and supply. We will evaluate LULC change
and use the persistent classes within 2001-2019 (\textgreater{} 80\%) to
analyze the zcNDVI index and its interconnection with precipitation
deficit, AED, ET, and vegetation cover. Finally, we will investigate if
the observed changes are linked to the TWS and SM.

\hypertarget{study-area}{%
\section{Study area}\label{study-area}}

\begin{figure}[H]
\centering
\includegraphics[width=\textwidth]{../output/figs/map_study_con_landcover.png}
\caption{ (\textbf{a}) Location of Central Chile and zones north (NCCH), central (CCCH), and south (SCCH) Central Chile. (\textbf{b}) Topography reference map. (\textbf{c}) Land cover classes for 2019. (\textbf{d}) Persistent land cover classess (> 80\%) for 2001-2019.}
\end{figure}

\hypertarget{materials-and-methods}{%
\section{Materials and Methods}\label{materials-and-methods}}

\hypertarget{satellite-data}{%
\subsection{Satellite data}\label{satellite-data}}

\hypertarget{zcndvi-and-cluster-zones}{%
\subsection{zcNDVI and cluster zones}\label{zcndvi-and-cluster-zones}}

\hypertarget{landcover-change-and-persistence}{%
\subsection{Landcover change and
persistence}\label{landcover-change-and-persistence}}

\hypertarget{vegetation-cover}{%
\subsection{Vegetation Cover}\label{vegetation-cover}}

\hypertarget{evapotranspiration-soil-moisture-and-water-storage}{%
\subsection{Evapotranspiration, Soil moisture and water
storage}\label{evapotranspiration-soil-moisture-and-water-storage}}

\hypertarget{data-analysis}{%
\subsection{Data analysis}\label{data-analysis}}

\hypertarget{results}{%
\section{Results}\label{results}}

\hypertarget{zcndvi-and-cluster-zones-1}{%
\subsection{zcNDVI and cluster zones}\label{zcndvi-and-cluster-zones-1}}

\hypertarget{landcover-change-and-persistent}{%
\subsection{Landcover Change and
persistent}\label{landcover-change-and-persistent}}

\begin{table}[!ht]

\caption{\label{tab:unnamed-chunk-1}Value of linear change trend next to time-series plot of surface, per landcover class (IGBP MCD12Q1.006) for 2001-2019 through Central Chile. Red dots on the plots indicate the maximum and minimum surface.}
\centering
\resizebox{\linewidth}{!}{
\begin{tabular}[t]{r>{}r>{}r>{}r>{}r>{}r>{}r}
\toprule
\multicolumn{1}{c}{ } & \multicolumn{6}{c}{Trend of change [$km^2 year^{-1}$]} \\
\cmidrule(l{3pt}r{3pt}){2-7}
zone & Shrubland & Savanna & Grassland & Barren land & Forest & Cropland\\
\midrule
norte grande & -2.3\includegraphics[width=0.67in, height=0.17in]{article_files/figure-latex//plot_d396c19134a3d.pdf} & 0.1\includegraphics[width=0.67in, height=0.17in]{article_files/figure-latex//plot_d396c11281a1c.pdf} & -31.2\includegraphics[width=0.67in, height=0.17in]{article_files/figure-latex//plot_d396c76e40d4c.pdf} & 32.8\includegraphics[width=0.67in, height=0.17in]{article_files/figure-latex//plot_d396cc832d9.pdf} & NA\includegraphics[width=0.67in, height=0.17in]{article_files/figure-latex//plot_d396c22c3df0c.pdf} & NA\includegraphics[width=0.67in, height=0.17in]{article_files/figure-latex//plot_d396c1e23c330.pdf}\\
norte chico & 79.3\includegraphics[width=0.67in, height=0.17in]{article_files/figure-latex//plot_d396c3e39b63c.pdf} & -66.1\includegraphics[width=0.67in, height=0.17in]{article_files/figure-latex//plot_d396c3a55945c.pdf} & -100.2\includegraphics[width=0.67in, height=0.17in]{article_files/figure-latex//plot_d396c5ea10cfa.pdf} & 104.0\includegraphics[width=0.67in, height=0.17in]{article_files/figure-latex//plot_d396c3e253181.pdf} & 0.0\includegraphics[width=0.67in, height=0.17in]{article_files/figure-latex//plot_d396c6e3f818e.pdf} & -13.1\includegraphics[width=0.67in, height=0.17in]{article_files/figure-latex//plot_d396c381a2e04.pdf}\\
zona central & 130.2\includegraphics[width=0.67in, height=0.17in]{article_files/figure-latex//plot_d396c752a96cf.pdf} & -128.6\includegraphics[width=0.67in, height=0.17in]{article_files/figure-latex//plot_d396c400c5e63.pdf} & 89.9\includegraphics[width=0.67in, height=0.17in]{article_files/figure-latex//plot_d396c511304bc.pdf} & 23.3\includegraphics[width=0.67in, height=0.17in]{article_files/figure-latex//plot_d396c58fe26f6.pdf} & -66.2\includegraphics[width=0.67in, height=0.17in]{article_files/figure-latex//plot_d396c16afaacc.pdf} & -24.4\includegraphics[width=0.67in, height=0.17in]{article_files/figure-latex//plot_d396c66febecf.pdf}\\
zona sur & -14.6\includegraphics[width=0.67in, height=0.17in]{article_files/figure-latex//plot_d396c1ee86964.pdf} & -316.2\includegraphics[width=0.67in, height=0.17in]{article_files/figure-latex//plot_d396c457bef18.pdf} & -55.9\includegraphics[width=0.67in, height=0.17in]{article_files/figure-latex//plot_d396c688683cf.pdf} & 2.1\includegraphics[width=0.67in, height=0.17in]{article_files/figure-latex//plot_d396c7f0e5ef0.pdf} & 412.4\includegraphics[width=0.67in, height=0.17in]{article_files/figure-latex//plot_d396c4f819a2c.pdf} & 30.8\includegraphics[width=0.67in, height=0.17in]{article_files/figure-latex//plot_d396c1671afb3.pdf}\\
zona austral & -44.6\includegraphics[width=0.67in, height=0.17in]{article_files/figure-latex//plot_d396c7a04eff2.pdf} & 163.9\includegraphics[width=0.67in, height=0.17in]{article_files/figure-latex//plot_d396c292ba1fa.pdf} & 226.1\includegraphics[width=0.67in, height=0.17in]{article_files/figure-latex//plot_d396c362a0ad3.pdf} & -80.2\includegraphics[width=0.67in, height=0.17in]{article_files/figure-latex//plot_d396c32966b1b.pdf} & -9.1\includegraphics[width=0.67in, height=0.17in]{article_files/figure-latex//plot_d396c211e7d1e.pdf} & -1.0\includegraphics[width=0.67in, height=0.17in]{article_files/figure-latex//plot_d396cb504520.pdf}\\
\bottomrule
\end{tabular}}
\end{table}

\hypertarget{spi-aed-and-et}{%
\subsection{SPI, AED, and ET}\label{spi-aed-and-et}}

\hypertarget{sm-and-total-storage}{%
\subsection{SM and Total Storage}\label{sm-and-total-storage}}

\hypertarget{discussion}{%
\section{Discussion}\label{discussion}}

Authors should discuss the results and how they can be interpreted in
perspective of previous studies and of the working hypotheses. The
findings and their implications should be discussed in the broadest
context possible. Future research directions may also be highlighted.

\hypertarget{conclusion}{%
\section{Conclusion}\label{conclusion}}

This section is not mandatory, but can be added to the manuscript if the
discussion is unusually long or complex.

\renewcommand\refname{References}
\bibliography{references.bib}


\end{document}
