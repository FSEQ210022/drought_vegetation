% Options for packages loaded elsewhere
\PassOptionsToPackage{unicode}{hyperref}
\PassOptionsToPackage{hyphens}{url}
\PassOptionsToPackage{dvipsnames,svgnames,x11names}{xcolor}
%
\documentclass[
  number,
  preprint,
  3p]{elsarticle}

\usepackage{amsmath,amssymb}
\usepackage{iftex}
\ifPDFTeX
  \usepackage[T1]{fontenc}
  \usepackage[utf8]{inputenc}
  \usepackage{textcomp} % provide euro and other symbols
\else % if luatex or xetex
  \usepackage{unicode-math}
  \defaultfontfeatures{Scale=MatchLowercase}
  \defaultfontfeatures[\rmfamily]{Ligatures=TeX,Scale=1}
\fi
\usepackage{lmodern}
\ifPDFTeX\else  
    % xetex/luatex font selection
\fi
% Use upquote if available, for straight quotes in verbatim environments
\IfFileExists{upquote.sty}{\usepackage{upquote}}{}
\IfFileExists{microtype.sty}{% use microtype if available
  \usepackage[]{microtype}
  \UseMicrotypeSet[protrusion]{basicmath} % disable protrusion for tt fonts
}{}
\makeatletter
\@ifundefined{KOMAClassName}{% if non-KOMA class
  \IfFileExists{parskip.sty}{%
    \usepackage{parskip}
  }{% else
    \setlength{\parindent}{0pt}
    \setlength{\parskip}{6pt plus 2pt minus 1pt}}
}{% if KOMA class
  \KOMAoptions{parskip=half}}
\makeatother
\usepackage{xcolor}
\setlength{\emergencystretch}{3em} % prevent overfull lines
\setcounter{secnumdepth}{5}
% Make \paragraph and \subparagraph free-standing
\ifx\paragraph\undefined\else
  \let\oldparagraph\paragraph
  \renewcommand{\paragraph}[1]{\oldparagraph{#1}\mbox{}}
\fi
\ifx\subparagraph\undefined\else
  \let\oldsubparagraph\subparagraph
  \renewcommand{\subparagraph}[1]{\oldsubparagraph{#1}\mbox{}}
\fi


\providecommand{\tightlist}{%
  \setlength{\itemsep}{0pt}\setlength{\parskip}{0pt}}\usepackage{longtable,booktabs,array}
\usepackage{calc} % for calculating minipage widths
% Correct order of tables after \paragraph or \subparagraph
\usepackage{etoolbox}
\makeatletter
\patchcmd\longtable{\par}{\if@noskipsec\mbox{}\fi\par}{}{}
\makeatother
% Allow footnotes in longtable head/foot
\IfFileExists{footnotehyper.sty}{\usepackage{footnotehyper}}{\usepackage{footnote}}
\makesavenoteenv{longtable}
\usepackage{graphicx}
\makeatletter
\def\maxwidth{\ifdim\Gin@nat@width>\linewidth\linewidth\else\Gin@nat@width\fi}
\def\maxheight{\ifdim\Gin@nat@height>\textheight\textheight\else\Gin@nat@height\fi}
\makeatother
% Scale images if necessary, so that they will not overflow the page
% margins by default, and it is still possible to overwrite the defaults
% using explicit options in \includegraphics[width, height, ...]{}
\setkeys{Gin}{width=\maxwidth,height=\maxheight,keepaspectratio}
% Set default figure placement to htbp
\makeatletter
\def\fps@figure{htbp}
\makeatother

\makeatletter
\makeatother
\makeatletter
\makeatother
\makeatletter
\@ifpackageloaded{caption}{}{\usepackage{caption}}
\AtBeginDocument{%
\ifdefined\contentsname
  \renewcommand*\contentsname{Table of contents}
\else
  \newcommand\contentsname{Table of contents}
\fi
\ifdefined\listfigurename
  \renewcommand*\listfigurename{List of Figures}
\else
  \newcommand\listfigurename{List of Figures}
\fi
\ifdefined\listtablename
  \renewcommand*\listtablename{List of Tables}
\else
  \newcommand\listtablename{List of Tables}
\fi
\ifdefined\figurename
  \renewcommand*\figurename{Figure}
\else
  \newcommand\figurename{Figure}
\fi
\ifdefined\tablename
  \renewcommand*\tablename{Table}
\else
  \newcommand\tablename{Table}
\fi
}
\@ifpackageloaded{float}{}{\usepackage{float}}
\floatstyle{ruled}
\@ifundefined{c@chapter}{\newfloat{codelisting}{h}{lop}}{\newfloat{codelisting}{h}{lop}[chapter]}
\floatname{codelisting}{Listing}
\newcommand*\listoflistings{\listof{codelisting}{List of Listings}}
\makeatother
\makeatletter
\@ifpackageloaded{caption}{}{\usepackage{caption}}
\@ifpackageloaded{subcaption}{}{\usepackage{subcaption}}
\makeatother
\makeatletter
\@ifpackageloaded{tcolorbox}{}{\usepackage[skins,breakable]{tcolorbox}}
\makeatother
\makeatletter
\@ifundefined{shadecolor}{\definecolor{shadecolor}{rgb}{.97, .97, .97}}
\makeatother
\makeatletter
\makeatother
\makeatletter
\makeatother
\journal{Journal Name}
\ifLuaTeX
  \usepackage{selnolig}  % disable illegal ligatures
\fi
\usepackage[]{natbib}
\bibliographystyle{elsarticle-num}
\IfFileExists{bookmark.sty}{\usepackage{bookmark}}{\usepackage{hyperref}}
\IfFileExists{xurl.sty}{\usepackage{xurl}}{} % add URL line breaks if available
\urlstyle{same} % disable monospaced font for URLs
\hypersetup{
  pdftitle={Drought assessment of the impact of water supply and demand over land cover in continental Chile for 2000-2023 from ERA5-Land and MODIS datasets},
  pdfauthor={Francisco Zambrano},
  pdfkeywords={drought, land cover change, satellite},
  colorlinks=true,
  linkcolor={blue},
  filecolor={Maroon},
  citecolor={Blue},
  urlcolor={Blue},
  pdfcreator={LaTeX via pandoc}}

\setlength{\parindent}{6pt}
\begin{document}

\begin{frontmatter}
\title{Drought assessment of the impact of water supply and demand over
land cover in continental Chile for 2000-2023 from ERA5-Land and MODIS
datasets}
\author[1]{Francisco Zambrano%
\corref{cor1}%
\fnref{fn1}}
 \ead{francisco.zambrano@umayor.cl} 

\affiliation[1]{organization={Universidad Mayor, Hémera Centro de
Observación de la Tierra, Facultad de Ciencias, Escuela de Ingeniería en
Medio Ambiente y Sustentabilidad},city={Santiago,
Chile},postcode={7500994},postcodesep={}}

\cortext[cor1]{Corresponding author}
\fntext[fn1]{This is the first author footnote.}
        
\begin{abstract}
Human-induced greenhouse gas emissions have increased the frequency
and/or intensity of weather and climate extremes. Central Chile has been
affected by a persistent drought which is impacting the hydrological
system and vegetation development. The region has been the focus of
research studies due to the diminishing water supply, this persistent
period of water scarcity has been defined as a ``mega drought''.
Nevertheless, our results evidence that the water deficit has expanded
beyond. Our goal is to analyze the impact of drought, measured by
drought indices of water supply/demand and vegetation status, in the
LULCC (land use land cover change) over continental Chile. For the
analysis, continental Chile was divided into five zones according to a
latitudinal gradient: ``Norte Grande'', ``Norte Chico'', ``Zona
Central'', ``Zona Sur'', and ``Zona Austral''. We used monthly climatic
re-analysis variables for precipitation, temperature and soil moisture
for 1981-2023 from ERA5-Land; and MODIS (Moderate-Resolution Imaging
Spectroradiometer) product MCD12Q1 for land cover for 2001-2021, and the
NDVI vegetation index from product MOD13A2 collection 6.1 for 2000-2023,
both from collection 6.1. We estimated atmospheric evaporative demand
(AED) by combining the Hargreaves-Samani equation with the ERA5-Land
temperature. We derived the drought indices SPI (Standardized
Precipitation Index), SPEI (Standardized Precipitation
Evapotranspiration Index), EDDI (Evaporative Demand Drought Index), zcSM
(standardized anomaly of cumulative soil moisture), and the zcNDVI
(standardized anomaly of cumulative NDVI). These indices were calculated
for time scales of 1, 3, 6, 12, 24, and 36 months, except for zcNDVI (1,
3, and 6 months). We analyzed the temporal correlation of SPI, SPEI,
EDDI, and zcSM with zcNDVI to have insights into the impact of water
supply and demand on vegetation. Our results showed that LULCC had an
increasing trend of 412 {[}km2yr−1{]} of forest expansion in the ``Zona
Sur'', together with a decreasing trend of 24 {[}km2yr−1{]} of cropland
contraction in the ``Zona Central'' meanwhile the ``Zona Sur'' showed an
increase of 31 {[}km2yr−1{]}, and a contraction of 80 {[}km2yr−1{]} of
bare soil in the ``Zona Austral''. The EDDI was the less correlated
index for the five macro zones and the five types of land cover, showing
that the temperature in Chile has little impact on vegetation. Higher
r-squared values, between 0.5 and 0.8, were obtained at ``Norte Chico''
and ``Zona Central'' for the land cover types of savanna, shrubland,
grassland, and croplands for the indices SPEI and zcSM at time scales of
12 and 24 months. The forest type reaches a r-squared of
\textasciitilde0.5 for zcSM of 12 months. The results indicate that the
``Norte Chico'' and ``Zona Central'' are the most sensitive regions to
water supply deficits longer than a year, potentially explained by a low
capacity of water storage in those zones that should be further
investigated.
\end{abstract}





\begin{keyword}
    drought \sep land cover change \sep 
    satellite
\end{keyword}
\end{frontmatter}
    \ifdefined\Shaded\renewenvironment{Shaded}{\begin{tcolorbox}[sharp corners, interior hidden, frame hidden, borderline west={3pt}{0pt}{shadecolor}, enhanced, breakable, boxrule=0pt]}{\end{tcolorbox}}\fi

\hypertarget{introduction}{%
\section{Introduction}\label{introduction}}

The sixth assessment report (AR6) of the IPCC \citep{IPCC2023} indicates
that human-induced greenhouse gas emissions have increased the frequency
and/or intensity of some weather and climate extremes, and the evidence
has been strengthened since AR5 \citep{IPCC2013}. There is high
confidence that increasing global warming can expand the land area
affected by increasing drought frequency and severity
\citep{IPCCCH112021}. Furthermore, drought increases tree mortality and
triggers changes in land cover and, consequently, land use, thus
impacting ecosystems \citep{Crausbay2017}. Nevertheless, there is a lack
of understanding of how the alteration in water supply and demand is
affecting land cover transformations.

Precipitation is the primary driver of drought, which impacts soil
moisture, hydrological regimes, and vegetation productivity. Initially,
drought was commonly classified as meteorological, hydrological, and
agricultural \citep{Wilhite1985}. Lately, \citep{Loon2016} and
\citep{AghaKouchak2021} have given an updated definition of drought for
the Anthropocene, suggesting that it should be considered the feedback
of humans' decisions and activities that drives the anthropogenic
drought. Even though it has been argued that those definitions do not
fully address the ecological dimensions of drought. \citep{Crausbay2017}
proposed the ecological drought definition as \emph{``an episodic
deficit in water availability that drives ecosystems beyond thresholds
of vulnerability, impacts ecosystem services, and triggers feedback in
natural and/or human systems''}. Moreover, many ecological studies have
misinterpreted how to characterize drought, for example, sometimes
considering ``dry'' conditions as ``drought'' \citep{Slette2019}. On the
other hand, the AR6 \citep{IPCC2023} states that even if global warming
is stabilized at 1.5°--2°C, many parts of the world will be impacted by
more severe agricultural and ecological droughts. Then, there is a
challenge in conducting drought research, especially its impact on
ecosystems.

Chile has been facing a persistent rainfall deficit for more than a
decade \citep{Garreaud2017}, which has impacted the hydrological system
\citep{Boisier2018} and consequently the vegetation development
\citep{Zambrano2023}. Current drought conditions have affected crop
yields in Central Chile. Highlighting the growing seasons 2007-2008 and
2008-2009 \citep[\citep{Zambrano2018}]{Zambrano2016}, which impacted an
extensive surface in {[}COMPLETAR{]}. But, in 2019--2020, the drought
intensity reached an extreme condition in North 34°S not seen for at
least 40 years \citep{Zambrano2023}, thus affecting forest, grassland,
and cropland areas. The prolonged lack of precipitation in Central Chile
is producing changes in ecosystem dynamics that must be studied.

Satellite remote sensing \citep{West2019, AghaKouchak2015} is the
primary method to evaluate how meteorological drought impacts vegetation
dynamics. Since the 90's multiple vegetation drought indices have been
developed, such as (VCI,\citep{Kogan1990}; TCI, \citep{Kogan1995};
zNDVI, \citep{Peters2002}; VegDri, \citep{Brown2008}) that have allowed
making spatiotemporal analysis. Although we can derive these indices for
any time during the year (depending on satellite revisit), they are more
relevant during the the growing season \citep{Mishra2015}. Although
modeling phenology is a complex task (e.g, \ldots.), satellites offer
strategies that help to address these issues
\citep{Younes2021, Vrieling2018, Cai2017}. Moreover, land cover dynamics
products such as the MCD12Q2 from the USGS \citep{Friedl2019} provide
phenology metrics. Rather than estimating these parameters at a given
point in time, some authors have proposed aggregating these indices
throughout the season. \citep{Meroni2017} proposes to accumulate the
fractional active photosynthetic active radiation (FAPAR) between the
start (SOS), and the end of the season (EOS) thus obtaining zCFAPAR as a
proxy for biomass productivity. \citep{Zambrano2018} used the same
approach but with the NDVI (Normalized Difference Vegetation Index),
deriving the zcNDVI for Central Chile. Besides, land use land cover
(LULC) change can be driven by drought \citep{Tran2019, Akinyemi2021}.
To analyze these changes, multiple time-series LULC products exist, such
as the MCD12Q1 \citep{Friedl2019} and the ESA CCI-LC (ESA 2017). The
LULC product, together with the vegetation drought index can help
evaluate the impact of drought on the ecosystem.

Vegetation drought indices are commonly used as proxies of productivity
\citep{Paruelo2016, Schucknecht2017}. The main environmental variables
that affect productivity are water supply (i.e., precipitation) and
demand (i.e., evapotranspiration)\citep{Mishra2015} commonly collected
from weather stations. In developing countries, such as Chile, gaps in
historical records are challenging. However, satellite products can be
used to fill these gaps.

To evaluate drought, the World Meteorological Organization (WMO;
\citep{WMO2012}) has proposed the Standardized Precipitation Index (SPI;
\citep{Mckee1993}), a multiscalar drought index that allows to monitor
precipitation deficit from short- to long-term. For water supply,
\citep{Zambrano2017} has already correlated i tInfraRed Precipitation
against weather Station data (CHIRPS; \citep{Funk2015}). On the other
hand, vegetation biomass productivity also strongly correlates to ET,
which in turns depends on the atmospheric evaporative demand (AED)
\citep{FAO66, Pereira2015, Allen2005}. Due to increasing trends in air
temperatures, AED is increasing, driving a rise in ET
\citep{IPCCCH112021}. But, it is not always true \citep{Milly2016}. For
example, regions where AET is highest have the lowest ET.

The MOD16 product \citep{Running2021} provides AET and ET satellite
estimates and has been used to derive drought indices \citep{Mu2013}.
Soil moisture (SM) is an Essential Climate Variable (ECV) that modulates
vegetative growth. The climate change initiative (CCI) from the European
Space Agency (ESA) delivers the ESA CCI SM product \citep{Dorigo2017}
(current version 6.1), which has been helpful to monitor drought
\citep{Zhang2019}. Besides, total water storage can be retrieved by the
Gravity Recovery and Climate Experiment (GRACE), which allows analyzing
water availability changes \citep{Ahmed2014, Ma2017}. The balance
between water demand and supply by remote sensing can be examined
against vegetation productivity, also derived from remote sensing.

We aim to analyze the impact of drought over different types of land
cover classes in continental Chile for 2000-2023 by combining
environmental variables of biomass productivity and water demand/supply
gathered from earth observation products. The specific objective for the
study are i) to calculate multi-scalar drought indices for water demand
and supply for 1981-2023, ii) to analyze the relationship of a proxy o
biomass (zcNDVI) with drought indices, iii) to evaluate LULC change for
2001-2021 and its relation with drought indices, and iv) to assess if
the observed changes in drought indices and land cover are linked to the
TWS.

\hypertarget{study-area}{%
\section{Study area}\label{study-area}}

\begin{figure}[ht]

{\centering \includegraphics[width=0.8\textwidth,height=\textheight]{../output/figs/map_study_con_landcover.png}

}

\caption{(a) Chile and the five zones ``norte grande'', ``norte chico'',
``zona central'', ``zona sur'', and ``zona austral''. (b) Topography
reference map. (c) Land cover classes for 2021. (d) Persistent land
cover classes (\textgreater{} 80\%) for 2001-2021.}

\end{figure}

\hypertarget{materials-and-methods}{%
\section{Materials and Methods}\label{materials-and-methods}}

\hypertarget{data}{%
\subsection{Data}\label{data}}

\hypertarget{earth-observation-data}{%
\subsubsection{Earth observation data}\label{earth-observation-data}}

\hypertarget{in-situ-data}{%
\subsubsection{in-situ data}\label{in-situ-data}}

\hypertarget{drought-indices-for-water-demand-and-supply}{%
\subsection{Drought indices for water demand and
supply}\label{drought-indices-for-water-demand-and-supply}}

\hypertarget{analysis-of-a-biomass-proxy-with-drought-indices-of-supply-and-demand}{%
\subsection{Analysis of a biomass proxy with drought indices of supply
and
demand}\label{analysis-of-a-biomass-proxy-with-drought-indices-of-supply-and-demand}}

\hypertarget{lulc-change-for-2001-2021-and-its-relation-with-water-supply-and-demand}{%
\subsection{LULC change for 2001-2021 and its relation with water supply
and
demand}\label{lulc-change-for-2001-2021-and-its-relation-with-water-supply-and-demand}}

!{[}Proportion of land cover class from the persistent land cover for
2001-2021 (\textgreater80\%) per macrozone.

\begin{figure}

\begin{minipage}[t]{0.47\linewidth}

{\centering 

\raisebox{-\height}{

\includegraphics[width=7cm,height=\textheight]{../output/figs/LC_pers80_per_macrozone.png}

}

}

\end{minipage}%
%
\begin{minipage}[t]{0.53\linewidth}

{\centering 

\raisebox{-\height}{

\includegraphics[width=8cm,height=\textheight]{../output/figs/table_var_landcover_macro.png}

}

}

\end{minipage}%

\caption{\label{fig-elephants}Proportion of land cover class from the
persistent land cover for 2001-2021 (\textgreater80\%) per macrozone.
The table on the left shows the linear change trend next to time-series
plot of surface, per land cover class (IGBP MCD12Q1.006) for 2001-2021
through the fives zones on continental Chile. Blue dots on the plots
indicate the maximum and red dots minimum surface reach.}

\end{figure}

\hypertarget{total-water-storage-and-drought-indices}{%
\subsection{Total water storage and drought
indices}\label{total-water-storage-and-drought-indices}}

Terrestrial Water Storage (TWS) is defined as the total amount of water
stored on land that includes any natural or artificial water bodies,
such as groundwater, soil moisture, rivers, lakes, snowpack, ice, and
biomass water \citep{Humphrey2023, Deng2023}. TWS changes evidence the
effects of multiple water fluxes on the hydrological cycle
\citep{Deng2023}. These are reflected in the temporal variation of
observations of the Earth's gravitational field
\citep{Abolafia2021, Sabzehee2023} and, in recent decades, the twin
Gravity Recovery and Climate Experiment (GRACE) satellites and their
Follow-On mission (GRACE-FO) have provided valuable results on globally
distributed TWS anomalies \citep{Tapley2019, Ferreira2023}. The GRACE
mission was launched in March 2002 and was operational until October
2017 \citep{Ramjeawon2022}. Then its GRACE Follow-On successor was
launched in May 2018 \citep{Landerer2020, Yin2022}. The information
provided by these satellites is used to construct monthly maps of the
Earth's average gravity field, providing details of the movement of
water masses or water mass anomaly estimates relative to the long-term
average gravity field \citep{Humphrey2023, Wahr2004}.

In this study, RL06.1\_V3 GRACE mascon (mass concentration) monthly
solutions were used, which are provided by the Jet Propulsion Laboratory
(JPL-M, https://grace.jpl.nasa.gov). Each GRACE Tellus monthly grid at
0.5 degrees represents the deviation of surface mass for that month
relative to a reference time average (2004-2009 baseline), which is
subtracted from all other monthly grids to provide terrestrial water
storage anomalies (TWSA) \citep{Ramjeawon2022, Yin2022}. This JPL-M
version of the data employs a Coastal Resolution Improvement (CRI)
filter that reduces signal leakage errors across coastlines
\citep{Wiese2019, Wiese2016}. Although the mascon solutions greatly
reduced leakage errors, a gain-factor, which is used to enhance the
spatial resolution, was applied to the dataset as recommended for
hydrological studies \citep{Ramjeawon2022, Yin2022}. The water
storage/height anomalies are given in Equivalent Water Height or
Thickness units (EWH, cm) \citep{Sabzehee2023}, and its temporal
resolution monthly is from April 2002 to May 2023.

\hypertarget{validation-of-era5-land-variables}{%
\subsection{Validation of ERA5-Land
variables}\label{validation-of-era5-land-variables}}

\hypertarget{results}{%
\section{Results}\label{results}}

\hypertarget{biomass-proxy-with-drought-indices-of-supply-and-demand}{%
\subsection{Biomass proxy with drought indices of supply and
demand}\label{biomass-proxy-with-drought-indices-of-supply-and-demand}}

\hypertarget{lulc-change-for-2001-2021-and-its-relation-with-water-supply-and-demand-1}{%
\subsection{LULC change for 2001-2021 and its relation with water supply
and
demand}\label{lulc-change-for-2001-2021-and-its-relation-with-water-supply-and-demand-1}}

\hypertarget{total-water-storage-tws-and-drought-indices}{%
\subsection{Total water storage (TWS) and drought
indices}\label{total-water-storage-tws-and-drought-indices}}

\hypertarget{validation-of-era5-land-variables-1}{%
\subsection{Validation of ERA5-Land
variables}\label{validation-of-era5-land-variables-1}}

\hypertarget{discussion}{%
\section{Discussion}\label{discussion}}

\hypertarget{conclusion}{%
\section{Conclusion}\label{conclusion}}


\renewcommand\refname{References}
  \bibliography{references.bib}


\end{document}
