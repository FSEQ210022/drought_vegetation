% Options for packages loaded elsewhere
\PassOptionsToPackage{unicode}{hyperref}
\PassOptionsToPackage{hyphens}{url}
\PassOptionsToPackage{dvipsnames,svgnames,x11names}{xcolor}
%
\documentclass[
  authoryear,
  preprint,
  3p,
  onecolumn]{elsarticle}

\usepackage{amsmath,amssymb}
\usepackage{iftex}
\ifPDFTeX
  \usepackage[T1]{fontenc}
  \usepackage[utf8]{inputenc}
  \usepackage{textcomp} % provide euro and other symbols
\else % if luatex or xetex
  \usepackage{unicode-math}
  \defaultfontfeatures{Scale=MatchLowercase}
  \defaultfontfeatures[\rmfamily]{Ligatures=TeX,Scale=1}
\fi
\usepackage{lmodern}
\ifPDFTeX\else  
    % xetex/luatex font selection
\fi
% Use upquote if available, for straight quotes in verbatim environments
\IfFileExists{upquote.sty}{\usepackage{upquote}}{}
\IfFileExists{microtype.sty}{% use microtype if available
  \usepackage[]{microtype}
  \UseMicrotypeSet[protrusion]{basicmath} % disable protrusion for tt fonts
}{}
\makeatletter
\@ifundefined{KOMAClassName}{% if non-KOMA class
  \IfFileExists{parskip.sty}{%
    \usepackage{parskip}
  }{% else
    \setlength{\parindent}{0pt}
    \setlength{\parskip}{6pt plus 2pt minus 1pt}}
}{% if KOMA class
  \KOMAoptions{parskip=half}}
\makeatother
\usepackage{xcolor}
\setlength{\emergencystretch}{3em} % prevent overfull lines
\setcounter{secnumdepth}{5}
% Make \paragraph and \subparagraph free-standing
\ifx\paragraph\undefined\else
  \let\oldparagraph\paragraph
  \renewcommand{\paragraph}[1]{\oldparagraph{#1}\mbox{}}
\fi
\ifx\subparagraph\undefined\else
  \let\oldsubparagraph\subparagraph
  \renewcommand{\subparagraph}[1]{\oldsubparagraph{#1}\mbox{}}
\fi


\providecommand{\tightlist}{%
  \setlength{\itemsep}{0pt}\setlength{\parskip}{0pt}}\usepackage{longtable,booktabs,array}
\usepackage{calc} % for calculating minipage widths
% Correct order of tables after \paragraph or \subparagraph
\usepackage{etoolbox}
\makeatletter
\patchcmd\longtable{\par}{\if@noskipsec\mbox{}\fi\par}{}{}
\makeatother
% Allow footnotes in longtable head/foot
\IfFileExists{footnotehyper.sty}{\usepackage{footnotehyper}}{\usepackage{footnote}}
\makesavenoteenv{longtable}
\usepackage{graphicx}
\makeatletter
\def\maxwidth{\ifdim\Gin@nat@width>\linewidth\linewidth\else\Gin@nat@width\fi}
\def\maxheight{\ifdim\Gin@nat@height>\textheight\textheight\else\Gin@nat@height\fi}
\makeatother
% Scale images if necessary, so that they will not overflow the page
% margins by default, and it is still possible to overwrite the defaults
% using explicit options in \includegraphics[width, height, ...]{}
\setkeys{Gin}{width=\maxwidth,height=\maxheight,keepaspectratio}
% Set default figure placement to htbp
\makeatletter
\def\fps@figure{htbp}
\makeatother

\usepackage{lineno}\linenumbers \usepackage{multirow} \usepackage{lscape} \newcommand{\blandscape}{\begin{landscape}} \newcommand{\elandscape}{\end{landscape}}
\makeatletter
\makeatother
\makeatletter
\makeatother
\makeatletter
\@ifpackageloaded{caption}{}{\usepackage{caption}}
\AtBeginDocument{%
\ifdefined\contentsname
  \renewcommand*\contentsname{Table of contents}
\else
  \newcommand\contentsname{Table of contents}
\fi
\ifdefined\listfigurename
  \renewcommand*\listfigurename{List of Figures}
\else
  \newcommand\listfigurename{List of Figures}
\fi
\ifdefined\listtablename
  \renewcommand*\listtablename{List of Tables}
\else
  \newcommand\listtablename{List of Tables}
\fi
\ifdefined\figurename
  \renewcommand*\figurename{Figure}
\else
  \newcommand\figurename{Figure}
\fi
\ifdefined\tablename
  \renewcommand*\tablename{Table}
\else
  \newcommand\tablename{Table}
\fi
}
\@ifpackageloaded{float}{}{\usepackage{float}}
\floatstyle{ruled}
\@ifundefined{c@chapter}{\newfloat{codelisting}{h}{lop}}{\newfloat{codelisting}{h}{lop}[chapter]}
\floatname{codelisting}{Listing}
\newcommand*\listoflistings{\listof{codelisting}{List of Listings}}
\makeatother
\makeatletter
\@ifpackageloaded{caption}{}{\usepackage{caption}}
\@ifpackageloaded{subcaption}{}{\usepackage{subcaption}}
\makeatother
\makeatletter
\@ifpackageloaded{tcolorbox}{}{\usepackage[skins,breakable]{tcolorbox}}
\makeatother
\makeatletter
\@ifundefined{shadecolor}{\definecolor{shadecolor}{rgb}{.97, .97, .97}}
\makeatother
\makeatletter
\makeatother
\makeatletter
\makeatother
\journal{Journal Name}
\ifLuaTeX
  \usepackage{selnolig}  % disable illegal ligatures
\fi
\usepackage[]{natbib}
\bibliographystyle{elsarticle-harv}
\IfFileExists{bookmark.sty}{\usepackage{bookmark}}{\usepackage{hyperref}}
\IfFileExists{xurl.sty}{\usepackage{xurl}}{} % add URL line breaks if available
\urlstyle{same} % disable monospaced font for URLs
\hypersetup{
  pdftitle={The effects of multi-dimensional drought on land cover change and vegetation productivity in continental Chile},
  pdfauthor={Francisco Zambrano; Anton Vrieling; Francisco Meza; Iongel Duran-Llacer; Francisco Fernández; Alejandro Venegas-González; Nicolas Raab; Dylan Craven},
  pdfkeywords={drought, land cover change, vegetation
productivity, satellite},
  colorlinks=true,
  linkcolor={blue},
  filecolor={Maroon},
  citecolor={Blue},
  urlcolor={Blue},
  pdfcreator={LaTeX via pandoc}}

\setlength{\parindent}{6pt}
\begin{document}

\begin{frontmatter}
\title{The effects of multi-dimensional drought on land cover change and
vegetation productivity in continental Chile}
\author[1,2]{Francisco Zambrano%
\corref{cor1}%
\fnref{fn1}}
 \ead{francisco.zambrano@umayor.cl} 
\author[3]{Anton Vrieling%
%
}
 \ead{a.vrieling@utwente.nl} 
\author[]{Francisco Meza%
%
}

\author[]{Iongel Duran-Llacer%
%
}

\author[4]{Francisco Fernández%
%
}
 \ead{francisco.fernandez@uss.cl} 
\author[]{Alejandro Venegas-González%
%
}

\author[]{Nicolas Raab%
%
}

\author[]{Dylan Craven%
%
}


\affiliation[1]{organization={Universidad Mayor, Hémera Centro de
Observación de la Tierra, Facultad de Ciencias, Escuela de Ingeniería en
Medio Ambiente y Sustentabilidad},city={Santiago,
Chile},postcode={7500994},postcodesep={}}
\affiliation[2]{organization={Universidad Mayor, Observatorio de Sequía
para la Agricultura y la Biodiversidad de Chile (ODES)},,postcodesep={}}
\affiliation[3]{organization={University of Twente, Faculty of
Geo-Information Science and Earth Observation (ITC)},city={Enschede, The
Netherlands},postcodesep={}}
\affiliation[4]{organization={Universidad San
Sebastian},,postcodesep={}}

\cortext[cor1]{Corresponding author}
\fntext[fn1]{This is the first author footnote.}







        
\begin{abstract}
The north-central region of Chile has been the focus of research studies
due to the persistent decrease in water supply, which is impacting the
hydrological system and vegetation development. This persistent period
of water scarcity has been defined as a mega-drought. The aim of our
study is to evaluate the interaction of drought, land cover change, and
vegetation productivity over continental Chile. We used drought indices
for atmospheric evaporative demand (AED), water supply, and soil
moisture from short-term (1, 3, 6 months) to long-term (12, 24, 36
months) time scales. We derived the drought indices using monthly
ERA5-Land reanalysis data spanning from 1981 to 2023. We utilized the
Moderate-Resolution Imaging Spectroradiometer (MODIS) datasets to derive
information on annual land cover and monthly vegetation productivity.
Our results showed that from south to north, Chile has a declining trend
in water supply, and across the whole country, there is an increase in
AED. These trends are stronger at longer time scales. The trend in
vegetation productivity in the north-central area is affecting, to a
higher degree, shrubland and savanna, followed by croplands and forests.
The drought explains about 30\% of the change in land cover type across
Chile for forest, grassland, shrubland, and savanna. The increase in AED
is the main driver of the change in land cover, followed by a reduction
in precipitation and soil moisture. The change in vegetation
productivity has been severe in the north-central part of the country
for all land cover types, particularly savanna, shrubland, and
croplands. The anomaly in soil moisture over the past 12 months (SSI-12)
is the main variable explaining these changes, followed by anomalies in
cumulated precipitation over one to two years (SPI-12 and SPI-24). Our
results provide insightful information that would help in developing
adaptation measures for ecosystems in Chile to cope with climate change
and drought.
\end{abstract}





\begin{keyword}
    drought \sep land cover change \sep vegetation productivity \sep 
    satellite
\end{keyword}
\end{frontmatter}
    \captionsetup{justification=raggedright,singlelinecheck=false}

\ifdefined\Shaded\renewenvironment{Shaded}{\begin{tcolorbox}[breakable, enhanced, borderline west={3pt}{0pt}{shadecolor}, frame hidden, boxrule=0pt, interior hidden, sharp corners]}{\end{tcolorbox}}\fi

\hypertarget{introduction}{%
\section{Introduction}\label{introduction}}

Drought is often classified as meteorological when there is a decrease
in precipitation below the mean average of several years (more than 30
years), hydrological when these anomalies last for long periods (months
to years) and affect water systems, and agricultural when the deficit
impacts plant health anomalies and leads to decreased productivity
\citep{Wilhite1985}. However, it is important to note that drought is
also influenced by human activities, which were not considered in the
definitions. Thus, \citet{Loon2016} and \citet{AghaKouchak2021} have
given an updated definition of drought for the Anthropocene, suggesting
that it should be considered the feedback of humans' decisions and
activities that drives the anthropogenic drought. Simultaneously,
drought leads to heightened tree mortality and induces alterations in
land cover and land use, ultimately affecting ecosystems
\citep{Crausbay2017}. Even though many ecological studies have
misinterpreted how to characterize drought, for example, sometimes
considering ``dry'' conditions as ``drought'' \citep{Slette2019}. Then,
\citet{Crausbay2017} proposed the ecological drought definition as ``an
episodic deficit in water availability that drives ecosystems beyond
thresholds of vulnerability, impacts ecosystem services, and triggers
feedback in natural and/or human systems.'' In light of current global
warming, it is crucial to study the interaction between drought and
ecosystems in order to understand their feedback and impact on water
security. \citep{Bakker2012}

Human-induced greenhouse gas emissions have increased the frequency
and/or intensity of drought as a result of global warming, according to
the sixth assessment report (AR6) of the Intergovernmental Panel on
Climate Change (IPCC) \citep{IPCC2023}. The evidence supporting this
claim has been strengthened since AR5 \citep{IPCC2013}. Recent studies,
however, have produced contrasting findings, suggesting that drought has
not exhibited a significant trend over the past forty years.
\citep{Vicente-Serrano2022, Kogan2020}. \citet{Vicente-Serrano2022}
analyzed the meteorological drought trend on a global scale, finding
that only in a few regions has there been an increase in the severity of
drought. Moreover, they attribute the increase in droughts over the past
forty years solely to an increase in atmospheric evaporative demand
(AED), which in turn enhances vegetation water demand, with important
implications for agricultural and ecological droughts. Also, they state
that ``the increase in hydrological droughts has been primarily observed
in regions with high water demand and land cover change''. Similarly,
\citet{Kogan2020} analyzed the drought trend using vegetation health
methods, finding that for the globe, hemispheres, and main
grain-producing countries, drought has not expanded or intensified for
the last 38 years. Further, \citet{IPCC2021} suggests that there is a
high degree of confidence that rising temperatures will increase the
extent, frequency, and severity of droughts. Also, AR6 \citep{IPCC2023}
predicts that many regions of the world will experience more severe
agricultural and ecological droughts even if global warming stabilizes
at 1.5°--2°C. To better evaluate the impact of drought trends on
ecosystems, assessments are needed that relate meteorological and soil
moisture variables to their effects on vegetation.

From 1960 to 2019, land use change has impacted around one-third of the
Earth's surface, which is four times more than previously thought
\citep{Winkler2021}. Multiple studies aim to analyze and forecast
changes in land cover globally \citep{Winkler2021, Song2018} and
regionally \citep{Chamling2020, Homer2020, Yang2021}. Some others seek
to analyze the impact of land cover change on climate conditions such as
temperature and precipitation \citep{Luyssaert2014, Pitman2012}. There
is less research on the interaction between drought and land cover
change \citep{Chen2022, Akinyemi2021, Peng2017}. \citet{Peng2017}
conducted a worldwide investigation utilizing net primary production to
examine the spatial and temporal variations in vegetation productivity
at global level. The study aimed to assess the influence of drought by
comparing the twelve-month Standardized Precipitation Evapotranspiration
Index (SPEI) and land cover change. According to their findings, drought
is responsible for 37\% of the decline in vegetation productivity, while
water availability accounts for 55\% of the variation. \citet{Chen2022}
studied the trend of vegetation greenness and productivity and its
relation to meteorological drought (SPEI of twelve months in December)
and soil moisture at the global level. The results showed lower
correlations (\textless0.2) for both variables. \citet{Akinyemi2021}
evaluates drought trends and land cover change using vegetation indices
in Botwsana in a semi-arid climate. These studies mostly looked at how
changes in land cover and vegetation productivity are related to a
single drought index (SPEI) over a single time period of 12 months. SPEI
takes into account the combined effect of precipitation and AED as a
water balance, but it does not allow us to know the contribution of each
variable on its own. Some things worth investigating in terms of land
cover change and vegetation productivity are: i) How do they respond to
short- to long-term meteorological and soil moisture droughts? ii) How
is the drought impacting land cover changes? And iii) How do they behave
in humid and arid climatic zones regarding drought? Likewise, there is a
lack of understanding of how the alteration in water supply and demand
is affecting land cover transformations.

For monitoring drought, the World Meteorological Organization recommends
the SPI (Standardized Precipitation Index) \citep{WMO2012}. The SPI is a
multi-scalar drought index that only uses precipitation to assess short-
to long-term droughts. The primary cause of drought is precipitation
anomalies, and temperature usually makes it worse \citep{Luo2017}.
Nowadays, there is an increase in attention toward using AED separately
to monitor droughts \citep{Vicente-Serrano2020}. One reason is due to
its attribution to increasing flash droughts in water-limited regions
\citep{Noguera2022}. \citet{Vicente-Serrano2010} proposed the
Standardized Precipitation Evapotranspiration Index (SPEI), which
incorporated the temperature effect by subtracting AED from
precipitation. SPEI allows for analysis of the combined effect of
precipitation and AED. Since its formulation, it has been used worldwide
for the study and monitoring of drought
\citep{Gebrechorkos2023, Liu2021}. \citet{Hobbins2016} and
\citet{McEvoy2016} developed the Evaporative Demand Drought Index (EDDI)
to monitor droughts solely using the AED, and it has proven effective in
monitoring flash droughts \citep{Li2024, Ford2023}. For soil moisture,
several drought indices exist, such as the Soil Moisture Deficit Index
(SDMI) \citep{Narasimhan2005} and the Soil Moisture Agricultural Drought
Index (SMADI) \citep{Souza2021}. \citet{Hao2013} and
\citet{AghaKouchak2014} proposed the Standardized Soil Moisture Index
(SSI), which has a similar formulation as the SPI, SPEI, and EDDI. Thus,
there are plenty of drought indices that allow for a comprehensive
assessment of drought on short- to long-term scales and that allow for
the use of single variables from the earth's water balance (e.g.,
precipitation, AED, soil moisture). The variation in climate variables
impacts vegetation development, and unfavorable conditions such as low
precipitation and high temperatures usually generate a decrease in
vegetation productivity. To monitor the response of vegetation, the
common practice is to use satellite data. The Normalized Difference
Vegetation Index (NDVI) has been widely used as a proxy for biomass
production \citep{Camps-Valls2021, Paruelo2016, Helman2014}. For Chile's
cultivated land, \citet{Zambrano2018} introduced the zcNDVI for
assessing seasonal biomass production in response to drought. Using this
information, we can advance our understanding of the impact of drought
on ecosystems.

Chile's diverse climatic and ecosystem types
\citep{Beck2023, Luebert2022} make it an ideal natural laboratory for
studying climate and ecosystems. Additionally, the country has
experienced severe drought conditions that have had significant effects
on vegetation and water storage. North-central Chile has faced a
persistent precipitation deficit since 2010, defined as a mega drought.
\citep{Garreaud2017}, which has impacted the Chilean ecosystem. This
megadrought was defined by the Standardized Precipitation Index (SPI) of
twelve months in December having values below one standard deviation.
Some studies have addressed how this drought affects single ecosystems
in terms of forest development \citep{Miranda2020, Venegas2018}, forest
fire occurrence \citep{UrrutiaJalabert2018}, and crop productivity
\citep{Zambrano2023, Zambrano2018, Zambrano2016}. We found one study
regarding land cover and drought in Chile. The study by
\citet{Fuentes2021} evaluates water scarcity and land cover change in
Chile between 29° and 39° of south latitude. \citet{Fuentes2021} used
the SPEI of one month for evaluating drought, which led to misleading
results. For example, they did not find a temporal trend in the SPEI but
found a decreasing trend in water availability and an increase trend on
AED, which in turn should have been capable of being captured with
longer time scales of the SPEI. The term ``megadrought'' in Chile is
used to describe a prolonged water shortage that lasts for several
years, resulting in a permanent deficit that impacts the hydrological
system \citep{Boisier2018}. Therefore, it is crucial to evaluate
temporal scales that consider the cumulative impact over a period of
several years. The association between drought and the environment in
Chile is not well comprehended. Hence, it is imperative to acquire a
more profound comprehension of the manner in which climatic and soil
moisture droughts influence environmental dynamics, in order to make
well-informed decisions on adaptation strategies.

Here, we analyze the multi-dimensional impacts of drought across
ecosystems in continental Chile. More specifically, we aim to assess: i)
short- to long-term temporal trends in multi-scalar drought indices; ii)
temporal changes in land-use cover and the direction and magnitude of
their relationships with trends in drought indices; and iii) the trend
in vegetation productivity and its relationship with drought indices
across Chilean ecosystems.

\hypertarget{study-area}{%
\section{Study area}\label{study-area}}

Continental Chile has diverse climate conditions with strong gradients
from north to south and east to west \citep{Aceituno2021}
(Figure~\ref{fig-studyArea} a), which determines its great ecosystem
diversity \citep{Luebert2022} (Figure~\ref{fig-studyArea} c). The Andes
Mountains are a main factor in climate latitudinal variation
\citep{Garreaud2009}. ``Norte Grande'' and ``Norte Chico'' predominate
in an arid desert climate with hot (Bwh) and cold (Bwk) temperatures. At
the south of ``Norte Chico,'' the climate changes to an arid steppe with
cold temperatures (Bsk). In these two northern regions, the land is
mostly bare, with a small surface of vegetation types such as shrubland
and grassland. In the zones ``Centro'' and the north half of ``Sur,''
the main climate is Mediterranean, with warm to hot summers (Csa and
Csb). Land cover in ``Centro'' comprises a significant amount of
shrubland and savanna (50\%), grassland (16\%), forest (8\%), and
croplands (5\%). An oceanic climate (Cfb) predominates in the south of
``Sur'' and the north of ``Austral.'' Those zones are high in forest and
grassland. The southern part of the country has a tundra climate, and in
``Austral,'' it is a cold semi-arid area with an extended surface of
grassland, forest, and, to a lesser extent, savanna.

\begin{figure*}[!ht]

{\centering \includegraphics{../output/figs/map_study_con_landcover.png}

}

\caption{\label{fig-studyArea}(a) Chile with the Koppen-Geiger climate
classes and the five macrozones ``Norte Grande'', ``Norte Chico'',
``Centro'', ``Sur'', and ``Austral''. (b) Topography reference map. (c)
land cover classes for 2022. (d) Persistent land cover classes
(\textgreater{} 80\%) for 2001-2022}

\end{figure*}

\hypertarget{materials-and-methods}{%
\section{Materials and Methods}\label{materials-and-methods}}

\hypertarget{data}{%
\subsection{Data}\label{data}}

\hypertarget{gridded-meteorological-and-vegetation-data}{%
\subsubsection{Gridded meteorological and vegetation
data}\label{gridded-meteorological-and-vegetation-data}}

To analyze land cover change, we use the classification scheme by the
IGBP (International Geosphere-Biosphere Programme) from the product
MCD12Q1 collection 6.1 from MODIS. The MCD12Q1 has a yearly frequency
from 2001 to 2022 and defines 17 classes. To derive a proxy for
vegetation productivity, we used the Normalized Difference Vegetation
Index (NDVI) from the product MOD13A3 collection 6.1 from MODIS
\citep{Didan2015}. MOD13A3 provides vegetation indices at 1km of spatial
resolution and monthly frequency. The NASA EOSDIS Land Processes
Distributed Active Archive Center (LP DAAC), USGS Earth Resources
Observation and Science (EROS) Center, Sioux Falls, South Dakota,
provided the MOD13A3 and MCD12Q1 from the online Data Pool, accessible
at https://lpdaa.usgs.gov/tools/data-pool/.

\begin{table}[!ht]
\caption{Description of the satellite and reanalysis data used}
\label{tab-desEOD}
\small
\centering
\begin{tabular}{p{0.13\textwidth}cp{0.3\textwidth}p{0.095\textwidth}ccc}
\hline
\multirow{1}{*}{\centering Product} & Sub-product & Variable & Spatial Resolution  & Period & Units & Short Name \\ 
\hline
\multirow{4}{*}{ERA5L} & ~ & Precipitation & \multirow{4}{*}{~0.1°} & \multirow{4}{*}{1981-2023} & mm & P \\ 
         &  & Maximum temperature & ~ & & $°C$ & $T_{max}$ \\ 
         &  & Minimum temperature & ~ & & $°C$ & $T_{min}$ \\ 
         &  & Volumetric Soil Water Content at 1m & ~ & & $m3/m3$ & SM \\ 
ERA5L* & & Atmospheric Evaporative Demand & 0.1° & 1981-2023 & mm & AED \\
        \multirow{2}{*}{MODIS} & MOD13A3.061 & Normalized Difference Vegetation Index & \multirow{2}{*}{~1 km} & 2000-2023 & ~ & NDVI \\ 
         & MCD12Q1.061 & land cover IGBP scheme & & 2001-2022 & ~ & land cover \\ 
\hline
\end{tabular}
{\raggedright *Calculated from maximum and minimum temperatures derived from ERA5L with Eq. \ref{eq-AED}. \par}
\end{table}

For soil moisture, water supply, and water demand variables, we used
ERA5L (ECMWF Reanalysis version 5 over land) \citep{MunozSabater2021}, a
reanalysis dataset that provides the evolution of atmospheric and land
variables since 1950. It has a spatial resolution of 0.1° (9 km), hourly
frequency, and global coverage. We selected the variables for total
precipitation, maximum and minimum temperature at 2 meters, and
volumetric soil water layers between 0 and 100cm of depth (layer 1 to
layer 3). Table \ref{tab-desEOD} shows a summary of the data and its
main characteristics.

\hypertarget{short--to-long-term-drought-trends}{%
\subsection{Short- to long-term drought
trends}\label{short--to-long-term-drought-trends}}

\hypertarget{atmospheric-evaporative-demand-aed}{%
\subsubsection{Atmospheric Evaporative Demand
(AED)}\label{atmospheric-evaporative-demand-aed}}

In order to compute the drought indices that use water demand, it is
necessary to first calculate the AED. To do this, we employed the
Hargreaves method \citep{Hargreaves1994, Hargreaves1985} by applying the
following equation:

\begin{equation}\protect\hypertarget{eq-AED}{}{AED = 0.0023\cdot Ra\cdot (T+17.8)\cdot (T_{max}-T_{min})^{0.5}}\label{eq-AED}\end{equation}

where \(Ra\) \((MJ\,m^2\, day^{-1})\) is extraterrestrial radiation;
\(T\), \(T_{max}\), and \(T_{min}\) are mean, maximum, and minimum
temperature \((°C)\) at 2m. For calculating \(Ra\) we used the
coordinate of the latitud of the centroid of each pixel. We chose the
method of Hargreaves to estimate AED because of its simplicity, which
only requires temperatures and extrarrestrial radiation. Also, it has
been recommended over other methods (e.g., Penman-Monteith) when the
access to climatic variables is limited \citep{Vicente-Serrano2014}.

\hypertarget{non-parametric-calculation-of-drought-indices}{%
\subsubsection{Non-parametric calculation of drought
indices}\label{non-parametric-calculation-of-drought-indices}}

To derive the drought indices of water supply and demand, soil moisture,
and vegetation (i.e., the proxy of productivity), we used the ERA5L
dataset and the MODIS product, with a monthly frequency for 1981--2023
and 2000--2023, respectively. The dought indices correspond to a
historical anomaly with regard to a variable (e.g., meteorological,
vegetation, or soil moisture). To account for the anomaly, the common
practice is to derive it following a statistical parametric methodology
in which it is assumed that the statistical distribution of the data is
known \citep{Heim2002}. A wrong decision is usually the highest source
of uncertainty \citep{Laimighofer2022}. In the case of Chile, due to its
high degree of climatic variability, it is complex to choose a proper
distribution without previous research. Here, we follow a non-parametric
methodology for the calculation of the drought indices, in a similar
manner as the framework proposed by \citet{Farahmand2015};
\citet{Hobbins2016};\citet{McEvoy2016}.

For the purpose of monitoring water supply drought, we used the
well-known Standardized Precipitation Index (SPI), which relies on
precipitation data. To evaluate water demand, we chose the Evaporative
Demand Drought Index (EDDI), developed by \citet{Hobbins2016} and
\citet{McEvoy2016}, which is based on the AED. The United States
currently monitors drought using the EDDI (https://psl.noaa.gov/eddi/)
as an experimental index. To consider the combined effect of water
supply and demand, we selected the SPEI \citep{Vicente-Serrano2010}. For
SPEI, an auxiliary variable \(D = P-AED\) is calculated. Soil moisture
is the main driver of vegetation productivity, particularly in semi-arid
regions \citep{Li2022}. Hence, for soil water drought, we used the SSI
(Standardized Soil Moisture Index) \citep{Hao2013, AghaKouchak2014}. In
our case, for the SSI, we used the average soil moisture from ERA5L at
1m depth. Finally, for the proxy of productivity, we used the zcNDVI
proposed by \citet{Zambrano2018}, which was derived from the monthly
time series of NDVI derived from MOD13A1. All the indices are
multi-scalar and can be used for the analysis of short- to long-term
droughts.

To derive the drought indices, first we must calculate the sum of the
variables with regard to the time scale (s). In this case, for
generalization purposes, we will use \(V\), referring to variables
\(P\), \(AED\), \(D\), \(NDVI\), and \(SM\) (Table \ref{tab-desEOD}). We
cumulated each \(V\) over the time series of \(n\) values (months), and
for the time scales \(s\):

\begin{equation}\protect\hypertarget{eq-sumvar}{}{A_{si} = \sum_{i=n-s-i+2}^{n-i+1} V_i\,\, \forall\, i\geq n-s+1  }\label{eq-sumvar}\end{equation}

The \(A_{si}\) corresponds to a moving window (convolution) that sums
the variable for time scales \(s\) from the last month, month by month,
until the first month in which it could sum for \(s\) months. An inverse
normal approximation \citep{Abramowitz1968} obtains the empirically
derived probabilities once the variable cumulates over time for the
scale \(s\). Then, we used the empirical Tukey plotting position
\citep{Wilks2011} over \(A_i\) to derive the \(P(a_i)\) probabilities
across a period of interest:

\begin{equation}\protect\hypertarget{eq-probPai}{}{P(A_i) = \frac{i-0.33}{n+0.33'}}\label{eq-probPai}\end{equation}

The drought indices \(SPI\), \(SPEI\), \(EDDI\), \(SSI\), and \(zcNDVI\)
are obtained following the inverse normal approximation:

\begin{equation}\protect\hypertarget{eq-DI}{}{DI(A_i) = W - \frac{C_0+C_1\cdot W + c_2 \cdot W^2}{1+d_1\cdot W +d_2\cdot W^2 +d_3\cdot W^3}}\label{eq-DI}\end{equation}

\(DI\) is referring to the drought index calculated for the variable
\(V\) (i.e., SPI, SPEI, EDDI, SSI, and zcNDVI). The values for the
constats are: \(C_0 = 2.515517\), \(C_1 = 0.802853\),
\(C_2 = 0.010328\), \(d_1 = 1.432788\), \(d_2 = 0.189269\), and
\(d3 = 0.001308\). For \(P(A) \leq 0.5\),
W=\(\sqrt{-2\cdot ln(P(A_i))}\) , and for \(P(A_i) > 0.5\), replace
\(P(A_i)\) with \(1-P(A_i)\) and reverse the sign of \(DI(A_i)\).

The drought indices were calculated for time scales of 1, 3, 6, 12, 24,
and 36 months at a monthly frequency for 1981--2023 in order to be used
for short- to long-term evaluation of drought. In the case of the proxy
of vegetation productivity (zcNDVI), it was calculated for a time scale
of six months at monthly frequency for 2000--2023. For zcNDVI, we test
time scales of 1, 3, 6, and 12 months in December and their correlation
with net primary production (NPP) obtained from the MOD17A3HGF product
from MODIS. We choose to use six months because r-squared with NPP
increases from one to six months and from six to 12 months has little
improvement. The r-squared were between 0.31 for forest and 0.72 for
shrubland (see supplementary material in Section S5).

\hypertarget{trend-of-drought-indices}{%
\subsubsection{Trend of drought
indices}\label{trend-of-drought-indices}}

To estimate if there are significant positive or negative trends for the
drought indices, we used the non-parametric test of Mann-Kendall
\citep{Kendall1975}. To determine the magnitude of the trend, we used
Sen's slope \citep{Sen1968}. Some of the advantages of applying this
methodology are that the Sen's slope is not affected by outliers as
regular regression does, and it is a non-parametric method that is not
influenced by the distribution of the data. We applied Mann-Kendall to
see if the trend was significant and Sen's slope to estimate the
magnitude of the trend. We did this to the six time scales from 1981 to
2023 (monthly frequency) and the indices SPI, EDDI, SPEI, and SSI. Thus,
we have trends per index and time scale (24 in total). Then, we
extracted the trend aggregated by macrozone and per land cover persitent
macroclasses.

\hypertarget{interaction-of-land-cover-and-drought}{%
\subsection{Interaction of land cover and
drought}\label{interaction-of-land-cover-and-drought}}

\hypertarget{land-cover-change}{%
\subsubsection{Land cover change}\label{land-cover-change}}

To analyze the land cover change, we use the IGBP scheme from the
MCD12Q1 collection 6.1 from MODIS. This product has been previously used
for studies of drought and land cover in Chile
\citep{Fuentes2021, Zambrano2018}. From the 17 classes, we regrouped
into ten macroclasses, as follows: classes 1-4 to forest, 5-7 to
shrublands, 8-9 to savannas, 10 as grasslands, 11 as wetlands, 12 and 14
to croplands, 13 as urban, 15 as snow and ice, 16 as barren, and 17 to
water bodies. Thus, we have a land cover raster time series with the ten
macroclasses for 2001 and 2023. We validate the land cover macroclasses
regarding a highly detailed (30 m of spatial resolution) land cover map
made for Chile by \citet{Zhao2016} for 2013-2014. Our results showed a
global accuracy of \textasciitilde0.82 and a F1 score of
\textasciitilde0.66. Section S2 in the Supplementary Material shows the
procedure for validation.

We calculated the surface occupied per land cover class into the five
macrozones (``Norte Grande'' to ``Austral'') per year for 2001--2023.
After that, we calculated the trend's change in surface per land cover
type and macroclass. We used Mann-Kendall for the significance of the
trend \citep{Kendall1975} and Sen's slope to calculate the magnitude
\citep{Sen1968}.

Later in this study, we will examine the variation in vegetation
productivity across various land cover types and how water demand and
supply, and soil moisture affect it. In order to avoid variations due to
a change in the land cover type from year-to-year that will wrongly
impact NDVI, we developed a persistence mask for land cover for
2001--2022. Thereby, we reduce an important source of variation on a
regional scale. Therefore, we generated a raster mask for IGBP MODIS per
pixel using macroclasses that remained unchanged for at least 80\% of
the years (2001--2022). This enabled us to identify regions where the
land cover macroclasses are persistent.

\hypertarget{relationship-between-land-cover-and-drought-trends}{%
\subsubsection{Relationship between land cover and drought
trends}\label{relationship-between-land-cover-and-drought-trends}}

We wanted to explore the relationship between the trend in land cover
classes and the trend in the drought indices. For this purpose, in order
to have more representative results, we conducted the analysis over
sub-basins within continental Chile. We use 469 basins, which have a
surface area between 0.0746 and ~24,000 (\(km^2\)), and a median area of
1,249 (\(km^2\)). For each basin, we calculate the relative trend per
land cover type, considering the proportion of the type relative to the
total surface of the basin. Then, we extracted per basin the average
trend of the drought indices SPI, SPEI, EDDI, SSI, and all their time
scales 1, 3, 6, 12, 24, and 36. Also, we extracted the average trend in
the proxy of vegetation productivity (zcNDVI). We wanted to analyze
which drought indices and time scales have a major impact on changes in
land cover type.

We have 25 predictors, including drought indices and vegetation
productivity. We analyzed the 25 predictors per type of landcover, thus
running six random forest models. Random forest uses multiple decision
trees and allows for classification and regression. Some advantages are
that it allows to find no linear relationship, reduces overfitting, and
allows to derive the variable importance. We used random forests for
regression and trained 1000 forests. To obtain more reliable results, we
resampled by creating ten folds, running a random forest per fold, and
calculating the r-squared (rsq), root mean square error (RMSE), and
variable importance.

The variable importance helps for a better understanding of the
relationships by finding which variable has a higher contribution to the
model. We calculate the variable's importance by permuting out-of-bag
(OOB) data per tree and computing the mean standard error in the OOB.
After permuting each predictor variable, we repeat the process for the
resting variable. We repeated this process ten times (per fold) to
obtain the performance metrics (rsq, RMSE, and variable importance).

\hypertarget{drought-impacts-on-vegetation-productivity}{%
\subsection{Drought impacts on vegetation
productivity}\label{drought-impacts-on-vegetation-productivity}}

We analyzed the trend of vegetation productivity over the unchanged land
cover macroclasses. To achieve this, we used the persistent mask of land
cover macroclasses. This way, we tried to reduce the noise in the
vegetation due to a change in land cover from year to year. We used the
zcNDVI as a proxy of vegetation productivity. In Chile's cultivated
land, \citet{Zambrano2018} introduced the zcNDVI for assessing seasonal
biomass production in relation to climate.

We examine the drought indices of water demand, water supply, and soil
moisture and their correlation with vegetation productivity. The
objective is to determine the impact of soil moisture and water demand
and supply on vegetation productivity. We want to address three main
questions: Which of the drought variables---supply, demand, or soil
moisture---most helps to explain the changes in plant productivity? How
do the short- to long-term time scales of the drought variable affect
vegetation productivity in Chile? And finally, how strong is the
relationship between the variables and the drought index? Thus, we will
be able to advance in understanding how climate is affecting vegetation,
considering the impact on the five land cover types: forest, cropland,
grassland, savanna, and shrubland.

We conducted an analysis on the linear correlation between the indices
SPI, SPEI, EDDI, and SSI over time periods of 1, 3, 6, 12, 24, and 36
months with zcNDVI. We used a method similar to that used by
\citet{Meroni2017} which compared the SPI time-scales with the
cumulative FAPAR (Fraction of Absorbed Photosynthetically Active
Radiation). We performed a pixel-to-pixel linear correlation analysis
for each index within the persistent mask of land cover macroclasses. We
first compute the Pearson coefficient of correlation for each of the six
time scales. A time scale is identified as the one that attains the
highest correlation (p \textless{} 0.05). We then extracted the Pearson
correlation coefficient corresponding to the time scales where the value
peaked. As a result, for each index, we generated two raster maps: 1)
containing the raster with values of the time scales and drought index
that reached the maximum correlation, and 2) having the magnitude of the
correlation obtained by the drought index at the time scales.

\hypertarget{software}{%
\subsection{Software}\label{software}}

For the downloading, processing, and analysis of the spatio-temporal
data, we used the open source software for statistical computing and
graphics, \texttt{R} \citep{R2023}. For downloading ERA5L, we used the
\texttt{\{ecmwfr\}} package \citep{Hufkens2019}. For processing raster
data, we used \texttt{\{terra\}} \citep{Hijmans2023} and
\texttt{\{stars\}} \citep{Pebesma2023}. For managing vectorial data, we
used \texttt{\{sf\}} \citep{Pebesma2018}. For the calculation of AED, we
used \texttt{\{SPEI\}} \citep{Bergueria2023}. For mapping, we use
\{tmap\} \citep{Tennekes2018}. For data analysis and visualization, the
suite \{tidyverse\} \citep{Wickham2019} was used. For the random forest
modeling, we used the \{tidymodels\}\citep{Kuhn2020} and
\{ranger\}\citep{Wright2017} packages.

\hypertarget{results}{%
\section{Results}\label{results}}

\hypertarget{short--to-long-term-drought-trends-1}{%
\subsection{Short- to long-term drought
trends}\label{short--to-long-term-drought-trends-1}}

Figure~\ref{fig-trendDI} shows the spatial variation of the trend for
the drought indices from short- to long-term scales. SPI and SPEI have a
decreasing trend from ``Norte Chico'' to ``Sur.'' However, there is an
increasing trend in ``Austral.'' The degree of the trend is stronger at
higher time scales. The SSI indicates that in ``Norte Grande,'' there
are surfaces that have increased in the soutwest part and in the
northeast have decreased, and is shown for all time scales. Similar to
SPI and SPEI, SSI decreases at higher time scales. EDDI showed a
positive trend for the whole of continental Chile, with a higher trend
toward the north and a descending gradient toward the south. The degree
of trend increases at higher time scales.

\blandscape

\begin{figure}

\begin{minipage}[t]{0.50\linewidth}

{\centering 

\raisebox{-\height}{

\includegraphics{../output/figs/trend_raster_SPI_1981-2023.png}

}

}

\subcaption{\label{fig-trendDI-1}SPI (Standardized Precipitation Index)}
\end{minipage}%
%
\begin{minipage}[t]{0.50\linewidth}

{\centering 

\raisebox{-\height}{

\includegraphics{../output/figs/trend_raster_SPEI_1981-2023.png}

}

}

\subcaption{\label{fig-trendDI-2}SPEI (Standardized Precipitation
Evapotranspiration Index)}
\end{minipage}%
\newline
\begin{minipage}[t]{0.50\linewidth}

{\centering 

\raisebox{-\height}{

\includegraphics{../output/figs/trend_raster_EDDI_1981-2023.png}

}

}

\subcaption{\label{fig-trendDI-3}EDDI (Evaporative Demand Drought
Index)}
\end{minipage}%
%
\begin{minipage}[t]{0.50\linewidth}

{\centering 

\raisebox{-\height}{

\includegraphics{../output/figs/trend_raster_zcSM_1981-2023.png}

}

}

\subcaption{\label{fig-trendDI-4}SSMI (Standardized Soil Moisture
Index)}
\end{minipage}%

\caption{\label{fig-trendDI}Linear trend of the drought index (*) at
time scales of 1, 3, 6, 12, 24, and 36 months for 1981-2023}

\end{figure}

\elandscape

\begin{figure}[!ht]

{\centering \includegraphics{../output/figs/trend_macrozone_drought_indices.png}

}

\caption{\label{fig-trendDIMacro}Trend per decade for the drought
indices SPI, EDDI, SPEI, and SSI aggregated by macrozone.}

\end{figure}

The Figure~\ref{fig-trendDIMacro} displays the averaged aggregation per
macrozone, the drought index, and the timescale. The macrozones that
reached the lowest trend for SPI, SPEI, and SSI are ``Norte Chico'' and
``Centro,'' where the indices also decrease at longer time scales.
Potentially explained due to the prolonged reduction in precipitation
that has affected the hydrological system in Chile. At 36 months, it
reaches trends between -0.03 and -0.04 (z-score) per decade for SPI,
SPEI, and SSI. For ``Sur,'' the behavior is similar, decreasing at
longer scales and having between -0.016 and -0.025 per decade for SPI,
SPEI, and SSI. ``Norte Grande'' has the highest trend at 36 months for
EDDI (0.042 per decade), and ``Centro'' has the lowest for SPI and SPEI.
In ``Norte Grande'' and ``Norte Chico,'' which are in a semi-arid
climate, it is evident that the EDDI has an effect on the difference
between the SPI and SPEI index, which is not seen in the other
macrozones. Contrary to the other macrozones, ``Austral'' showed an
increase in all indices, being the highest for EDDI at 36 months (0.025)
and the lowest for SSI, which shows only a minor increase in the trend.

\hypertarget{interaction-of-land-cover-and-drought-1}{%
\subsection{Interaction of land cover and
drought}\label{interaction-of-land-cover-and-drought-1}}

\hypertarget{land-cover-change-1}{%
\subsubsection{Land cover change}\label{land-cover-change-1}}

\begin{table}[!ht]
\caption{Surface of the land cover class that persist during 2001-2022}
\label{tab-landcoverSurf}
\includegraphics[width = .5\textwidth]{../output/figs/table_surface_landcover_macrozone.png}
\end{table}

\begin{figure}[!ht]

{\centering \includegraphics{../output/figs/LC_pers80_per_macrozone.png}

}

\caption{\label{fig-LCprop}Proportion of land cover class from the
persistent land cover for 2001-2022 (\textgreater80\%) per macrozone}

\end{figure}

For vegetation, we obtained and use hereafter five macroclasses of land
cover from IGBP MODIS: forest, shrubland, savanna, grassland, and
croplands. Figure~\ref{fig-studyArea}c shows the spatial distribution of
the macroclasses through Chile for the year 2022.
Figure~\ref{fig-studyArea}d shows the macroclasses of land cover
persistance (80\%) during 2021--2022, respectively (Table
\ref{tab-landcoverSurf}). Within continental Chile, barren land is the
land cover class with the highest surface area (277,870 \(km^2\)). The
largest type of vegetation, with 137,085 \(km^2\), is forest. Grassland
has 74,247 \(km^2\), savanna 55,206 \(km^2\), shrubland 25,341 \(km^2\),
and cropland 3,146 \(km^2\) (Table \ref{tab-landcoverSurf}). The
macrozones with major changes for 2001--2022 were ``Centro,'' ``Sur,''
and ``Austral,'' with 36\%, 31\%, and 34\% of their surface changing the
type of land cover, respectively (Figure~\ref{fig-studyArea} and Table
\ref{tab-landcoverTrend}). Figure~\ref{fig-LCprop} shows the summary of
the proportion of surface per land cover class and macrozone, derived
from the persistance mask over continental Chile.

\begin{table}[!ht]
\caption{The value of Sen's slope trend next to the time-series plot of surface per land cover class (IGBP MCD12Q1.016) for 2001–2022 through Central Chile. Values of zero indicate that there was not a significant trend. Red dots on the plots indicate the maximum and minimum values of surface.}
\label{tab-landcoverTrend}
\includegraphics[]{../output/figs/table_var_landcover_macro.png}
\end{table}

From the trend analysis in Table \ref{tab-landcoverTrend}, we can
indicate that the ``Norte Chico'' shows an increase in barren land of
111 \(km^2 yr^{-1}\) and a reduction in the class savanna of 70
\(km^2 yr^{-1}\). In the ``Centro'' and ``Sur,'' there are changes with
an important reduction in savanna (136 to 318 \(km^2 yr^{-1}\) ), and an
increase in shrubland and grassland. Showing a change for more dense
vegetation types. The area under cultivation (croplands) appears to be
shifting from the ``Centro'' to the ``Sur.'' Also, there is a high
increase in forest (397 \(km^2 yr^{-1}\) ) in the ``Sur,'' seemingly
replacing the savanna lost (Table \ref{tab-landcoverTrend}).

\hypertarget{relationship-between-drought-indices-and-land-cover-change}{%
\subsubsection{Relationship between drought indices and land cover
change}\label{relationship-between-drought-indices-and-land-cover-change}}

\begin{table}[!ht]
\caption{The five most important trends of drought indices in estimating the landcover trend per land cover type and the r-squared (rsq) reached by each random forest model.}
\label{tab-landcoverTrendRF}
\includegraphics[]{../output/figs/table_importance_trends_landcover_vs_drought.png}
\end{table}

According to Table \ref{tab-landcoverTrendRF}, the random forest models
for estimating the landcover trend from the trends in drought indices
reach an r-squared between 0.32 and 0.39 for the types of forest,
grassland, savanna, shrubland, and barren land. It is more likely that
short- and medium-term increases in AED (EDDI-6 and EDDI-12) and
short-term precipitation deficits (SPI-6 and SPEI-6) contributed to
changes in grassland and bare land. The short-term increase of AED
(EDDI-3 and EDDI-6) and the longer duration of the precipitation deficit
(SPI-24, SPI-36, and SPEI-36) most likely contribute to the changes in
shrubland. The changes in savanna are associated with a short- and
long-term increase in AED and a three-year precipitation deficit
(SPI-36). The increase in cumulative AED from 12 to 36 months is the
strongest associated variable that contributes to changes in forests,
followed by the reduction of soil moisture over six and 36 months. The
supplementary material in Section S3 provides further details about the
variable's importance.

\begin{figure}[!ht]

{\centering \includegraphics{../output/figs/points_landcover_drought_indices_trend_and_time_scale.png}

}

\caption{\label{fig-TrendsLandDrought}Relationship between the trend in
land cover change (y-axis) and the trend in drought indices (x-axis) for
the five macrozones. Vertical panels correspond to 1, 3, 6, 12, 24, and
36 months of the time scale by drought index. Horizontal panels show
each drought index}

\end{figure}

We study the connection between the SPI, EDDI, and SSI drought indices
and changes in land cover in Figure~\ref{fig-TrendsLandDrought}. To do
this, we compare the relative changes in land cover (in terms of the
total surface area per land cover type and macrozone) over six and
thirty-six months. Figure~\ref{fig-TrendsLandDrought} shows that the
forest in the ``Sur,'' shrubland and grassland in ``Centro,'' barren
land in ``Norte Chico,'' and savanna in ``Austral'' showed an increase
in the surface of landcover associated with an increase in EDDI. Savanna
in ``Centro,'' ``Sur,'' and ``Norte Chico'' decreases with the increase
in EDDI. The SPI and SSI showed similar behavior regarding the trend in
land cover type. A decrease in SPI and SSI is associated with an
increase in the surface in shurubland and grassland in ``Centro,''
forest in ``Sur,'' and barren land in ``Norte Chico,'' as well as a
decrease trend in savanna in ``Norte Chico,'' ``Centro,'' and ``Sur.''

\hypertarget{drought-impacts-on-vegetation-productivity-within-land-cover}{%
\subsection{Drought impacts on vegetation productivity within land
cover}\label{drought-impacts-on-vegetation-productivity-within-land-cover}}

\hypertarget{trends-in-vegetation-productivity}{%
\subsubsection{Trends in vegetation
productivity}\label{trends-in-vegetation-productivity}}

\begin{figure}[!ht]

{\centering \includegraphics{../output/figs/temporal_variation_zcNDVI6_macrozonas_con_mapa.png}

}

\caption{\label{fig-zcNDVI_var}(a) Map of the linear trend of the index
zcNDVI for 2000--2023. Greener colors indicate a positive trend; reder
colors correspond to a negative trend and a decrease in vegetation
productivity. Grey colors indicate either no vegetation or a change in
land cover type for 2001--2022. (b) Temporal variation of zcNDVI
aggregated at macrozone level within continental Chile. Each horizontal
panel corresponds to a macrozone from `Norte Grande' to `Austral'.}

\end{figure}

The temporal variation within the macrozones is shown in
Figure~\ref{fig-zcNDVI_var}b). There is a negative trend in ``Norte
Chico'' with -0.035 and ``Centro'' with -0.02 per decade. Vegetation
reached its lowest values for 2019-2022, with an extreme condition in
early 2020 and 2022 in the ``Norte Chico'' and ``Centro''. The ``Sur''
and ``Austral'' show a positive trend of around 0.012 and 0.016,
respectively, per decade (Figure~\ref{fig-zcNDVI_var}).

In Figure~\ref{fig-zcNDVI_var} it is showed the spatial map of trends in
zcNDVI (Figure~\ref{fig-zcNDVI_var}a). In ``Norte Grande,'' vegetation
productivity, as per the z-index, exhibits a yearly increase of 0.027
for grassland and 0.032 for shrubland. In the ``Norte Chico,'' savanna
has the lowest trend of -0.062, cropland -0.047, shrubland -0.042, and
grassland -0.037. In ``Centro,'' shrubland reaches -0.07, savanna
-0.031, cropland -0.024, forest -0.017, and grassland -0.005 per decade.
This decrease in productivity could be associated either with a
reduction in vegetation surface, a decrease in biomass, or browning.

\hypertarget{correlation-between-vegetation-productivity-and-drought-indices}{%
\subsubsection{Correlation between vegetation productivity and drought
indices}\label{correlation-between-vegetation-productivity-and-drought-indices}}

\begin{figure}[!ht]

{\centering \includegraphics{../output/figs/mapa_cor_selec_indices_zcNDVI6.png}

}

\caption{\label{fig-corTimeScale}Time scales per drought index that
reach the maximum coefficient of determination}

\end{figure}

\begin{figure}[!ht]

{\centering \includegraphics{../output/figs/mapa_cor_r_indices_zcNDVI6.png}

}

\caption{\label{fig-corPerson}Pearson correlation value for the time
scales and drought index that reach the maximum coefficient of
determination}

\end{figure}

Figure~\ref{fig-corTimeScale} shows the highest coefficient of
determination (r-squared, or rsq) found in the regression analysis
between zcNDVI and different drought indicators over time scales of 1,
3, 6, 12, 24, and 36 months. The spatial variation of time scales
reached per index is mostly for time scales above 12 months. In the case
of SSI, the predominant scales are 6 and 12 months. For all indices, to
the north, the time scales are higher and diminish toward the south
until the south part of ``Austral,'' where they increase. In
Figure~\ref{fig-corPerson}, the map of Pearson correlation values (r) is
shown. The EDDI reached correlations above 0.5 between ``Norte Chico''
and ``Sur.'' The correlation changes from negative to positive toward
the Andes Mountains and to the sea, just as in the northern part of
``Austral.'' The SPI and SPEI have similar results, with the higher
values in ``Norte Chico'' and ``Centro'' being higher than 0.6.
Following a similar spatial pattern as EDDI but with an opposite sign.
The SSI showed to be the index that has a major spatial extension with a
higher correlation. It has a similar correlation to SPI and SPEI for
``Norte Chico'' and ``Sur,''~ but has improvements for ``Sur.''

\begin{table}[!ht]
\caption{Summarry per land cover macroclass and macrozone regarding the correlation between zcNDVI with the drought indices EDDI, SPI, SPEI, and SSI for time scales of 1, 3, 6, 12, 24, and 36. The numbers in each cell indicate the time scale that reached the maximum correlation for the land cover and macrozone, and the color indicates the strength of the r-squared obtained with the index and the time scale.}
\label{tab-corlandcover}
\includegraphics[]{../output/figs/tabla_r_cor_macro_indice.png}
\end{table}

In Table \ref{tab-corlandcover}, we aggregate per macrozone and
landcover the correlation analysis presented in
Figure~\ref{fig-corTimeScale} and Figure~\ref{fig-corPerson}. According
to what is shown, forests seem to be the most resistant to drought.
Showing that only ``Centro'' is slightly (rsq = 0.25) impacted by a
12-month soil moisture deficit (SSI-12). In the ``Norte Chico'' and to a
lesser extent in the ``Norte Grande,'' it is evident that a SSI-12 with
a rsq = 0.45 and a decrease in water supply (SPI-36 and SPEI-24 with rsq
= 0.28 and 0.34, respectively) have an impact on grasslands. However,
this type was unaffected by soil moisture, water supply, or demand in
macrozones further south. The types that show to be most affected by
variation in climate conditions are shrublands, savannas, and croplands.
For savannas in ``Norte Chico,'' the SSI-12 and SPI-24 reached an rsq of
0.74 and 0.58, respectively. This value decreases to the south, but the
SSI-12 is still the variable explaining more of the variation in
vegetation productivity (rsq = 0.45 in ``Centro'' and 0.2 in ``Sur'').
In the case of croplands, the SPEI-12, SPI-36, and SSI-12 explain
between 45\% and 66\% of the variability in ``Norte Chico.'' The type of
land most impacted by climatic variation was shrubland, where soil
moisture explained 59\% and precipitation, 37\%, in ``Norte Chico'' and
``Centro,'' with SSI-12 being the most relevant variable, then SPI-36 in
``Norte Chico'' and SPI-24 in ``Sur.''

\hypertarget{discussion}{%
\section{Discussion}\label{discussion}}

\hypertarget{the-main-drivers-of-drought-in-chile}{%
\subsection{The main drivers of drought in
Chile}\label{the-main-drivers-of-drought-in-chile}}

\citet{Vicente-Serrano2022}, in a study at the global scale of drought
trends, indicates that there have not been significant trends in
meteorological drought since 1950. Also, state that the increase in
hidrological trend in some parts of the globe (northeast Brazil and the
Mediterranean region) is related to changes in land cover and
specifically to the rapidly increasing irrigated area, which
consequently increases water extraction. \citet{Kogan2020} analyzed the
agricultural drought impact globally and in the main grain producer
countries, finding that ``since 1980, the Earth warming has not changed
the drought area or intensity.'' In our study, we took into account the
variation in vegetation productivity in Chile, specifically in areas
without any changes in land cover, to prevent any misleading conclusions
about the increase in water demand due to land cover change. Our results
show a contrasting perspective. The SPI, SPEI, and SSI (water supply)
showed a decrease in trends, except for the southern part, and an
increase in EDDI (water demand). The trend, positive or negative, was
stronger as the time scales increased. Trends in the long term (e.g., 36
months) are evidence of how human-induced climate change is affecting
Chile, which seems to be due to an intense hydrological drought
resulting from the persistence of the precipitation deficit. We found
that there has been a significant trend in the decline of vegetation
productivity (zcNDVI) since 2000 for the north-central part of the
country, which has been extreme between 2020 and 2022 and has impacted
natural and cultivated land. Additionally, we demonstrated that the
drought, primarily due to an increase in AED, accounts for about 30\% of
the changes in land cover types (excluding croplands). These changes are
associated with a decrease in water demand from vegetation. Moreover,
the most water-demanding type, cropland, showed a decrease in the
north-central region, while barren land showed an increase. The
north-central part of the country primarily experienced these changes
due to a higher increase in AED. Thus, we have evidence of a significant
decline in water supply and an increase in AED for the north-central
part of Chile, which show to be the most relevant variables for drought
conditions. Some questions arise regarding what is occurring with the
cultivated land. We used the unchanged land cover to ensure that an
increase in surface area is not considered in the trend analysis. For
croplands, it could happen that some areas have changed the types of
crops for others with higher water demand, which consequently increases
water demand. However, this effect should be minor compared to the
decrease in water supply and increase in water demand at this scale of
analysis.

This shows that the main cause of the hydrological drought in Chile was
a steady drop in water supply made worse by an increase in AED, but it
seems that in zones most affected by drought, the main cause is not an
increase in vegetation water demand due to an intensification of
cultivated land (e.g., an increase in irrigated crops). North-central
Chile has experienced a decline in vegetation productivity across land
cover types, which is primarily attributable to variations in water
supply and soil moisture. An increase in water demand, such as an
increase in the surface area of irrigated crops, could strengthen this
trend. But it is out of the scope of this study. Future work should
focus on the regions where the drought has been more severe and has a
high proportion of irrigated crops to get insight on the real impact of
irrigation on those zones.

\hypertarget{land-cover-sensitivity-to-drought}{%
\subsection{Land cover sensitivity to
drought}\label{land-cover-sensitivity-to-drought}}

We analyzed two main impacts of drought on land cover. First, the
attribution of drought to the change in surface area per land cover
type. Drought accounts for about 30\% of the surface change per land
cover type, with the exception of croplands. The main variables
associated with these changes are the increase in AED and, in second
place, the decrease in precipitation. Second, we analyzed the time
series of drought indices and vegetation productivity per land cover
type. In this case, the most important variables that had an impact on
zcNDVI were the soil moisture deficit, followed by the precipitation
deficit, and in third place, AED.

In a study in the Yangtze River Basin in China, \citet{Jiang2020}
analyzed the impact of drought on vegetation using the SPEI and the
Enahanced Vegetation Index (EVI). They found that cropland was more
sensitive to drought than grassland, showing that cropland responds
strongly to short- and medium-term drought (\textless{} SPEI-6). In our
case, the SPEI-12 was the one that most impacted the croplands in
``Norte Chico'' and ``Centro.'' In general, most studies show that
croplands are most sensitive to short-term drought (\textless{} SPI-6)
\citep{Zambrano2016, Potopova2015, Dai2020, Rhee2010}. Short-term
precipitation deficits impact soil water, and thus less water is
available for plant growth. However, we found that in ``Norte Chico,''
an SPI-36 and SPEI-12 had a higher impact, which are associated with
hydrological drought (long-term), and in ``Centro,'' an SPI-12 and
SPEI-12. Thus, we attribute this impact to the hydrological drought that
has decreased groundwater storage \citep{Taucare2024}, which in turn is
impacted by long-term deficits, and consequently, the vegetation is more
dependent on groundwater. In ``Sur'' and ``Austral,'' the correlations
between drought indices and vegetation productivity decrease, as do the
time scales that reach the maximum r-squared. The possible reason for
this is that the most resistant types, forest and grassland, predominate
south of ``Centro.'' Also, drought episodes have been less frequent and
intense. The drought episodes have had a lower impact on water
availability for vegetation.

According to \citet{Senf2020}, severe drought conditions in Europe are a
significant cause of tree mortality. However, we discovered that
forests, as the most resilient land cover class to drought, experience
less variation in drought indices. Supporting this is
\citet{Fathi-Taperasht2022}, who asserts that Indian forests are the
most drought-resistant and recover rapidly. Similarly, the work of
\citet{Wu2024}, who analyzed vegetation loss and recovery in response to
meteorological drought in the humid subtropical Pearl River basin in
China, indicates that forests showed higher drought resistance. Using
Vegetation Optical Depth (VOD), kNDVI, and EVI, \citet{Xiao2023} tests
the resistance of ecosystems and finds that ecosystems with more forests
are better able to handle severe droughts than croplands. They attribute
the difference to a deeper rooting depth for trees, a higher water
storage capacity, and different water use strategies between forest and
cropland \citep{Xiao2023}. In contrast, \citet{Venegas2022}, who studied
Cryptocarya alba and Beilschmiedia miersii (both from the Lauraceae
family) that live in sclerophyllous forests in Chile, found that the
trees' overall growth had slowed down. This could mean that the natural
dynamics of their forests have changed. They attributed it to the
cumulative effects of the unprecedented drought (i.e., hydrological
drought).

Thus, we attribute that forest to being the most resistant to drought,
due to the fact that most of the species comprising it are highly
resilient to water scarcity compared to the other land cover classes.
Nonetheless, if we want to go deep in our analysis, we should use earth
observation data that is able to capture a higher level of detail. For
example, when we used MOD13A3 with a 1km spatial resolution to measure
vegetation condition, it took the average condition of 1 square
kilometer. Then, to use remote sensing to look at how a certain type of
forest (like sclerophyllous forest) changes in response to drought on a
local level, we should use operational products with higher spatial
resolutions, like those from Landsat or Sentinel. This will let us do a
more thorough analysis.

\hypertarget{vegetation-productivity-and-drought.}{%
\subsection{Vegetation productivity and
drought.}\label{vegetation-productivity-and-drought.}}

We found that the 12-month soil moisture deficit most affects the
productivity of vegetation in all land cover types along Chile. The main
external factors that affect biomass production by vegetation are actual
evapotranspiration and soil moisture, and the rate of ET in turn depends
on the availability of water storage in the root zone. Thus, soil
moisture plays a key role in land carbon uptake and, consequently, in
the production of biomass \citep{Humphrey2021}. Moreover,
\citet{Zhang2022} indicate there is a bidirectional causality between
soil moisture and vegetation productivity. Lastly, some studies have
redefined agricultural drought as soil moisture drought from a
hydrological perspective \citep{Loon2016, Samaniego2018}. Even though
soil moisture is the external factor most determinant of vegetation
biomass, there are multiple internal factors, such as species,
physiological characteristics, and plant hydraulics, that would affect
vegetation productivity. Because of that, we believe that agricultural
drought, referring to the drought that impacts vegetation productivity,
is the most proper term, as originally defined by \citet{Wilhite1985}.

The study results showed that the soil moisture-based drought index
(SSI) was better at explaining vegetation productivity across land cover
macroclasses than meteorological drought indices like SPI, SPEI, and
EDDI. In the early growing season and especially in irrigated rather
than rainfed croplands, soil moisture has better skills than SPI and
SPEI for estimating gross primary production (GPP). This according to
\citet{Chatterjee2022} evaluation of the SPI and SPEI and their
correlation with GPP in the CONUS. Also, \citet{Zhou2021} indicate that
the monthly scaled Standardized Water Deficit Index (SWDI) can
accurately show the effects of agricultural drought in most of China.
\citet{Nicolai2017} also looked at the time-lag between the SWDI and the
Vegetation Condition Index (VCI). They found that there was little to no
time-lag in croplands but a greater time-lag in forests.

In our case, there is strong spatial variability throughout Chile and
between classes, mainly attributable to climate heterogeneity,
hydrological status, or vegetation resistance to water scarcity. The
semi-arid ``Norte Chico'' and the Mediterranean ``Centro'' were where
SSI had the best performance. In Chile, medium-term deficits of 12
months are more relevant in the response of vegetation, which decreases
to the south, and in the case of croplands, they seem to react in a
shorter time, with six months (SSI-6) in ``Centro.'' This variation for
croplands could be related to the fact that in ``Norte Chico,'' the
majority of crops are irrigated, but to the south there is a higher
proportion of rainfed agriculture, which is most dependent on the
short-term availability of water. Rather, in the ``Norte Chico,'' the
orchards are more dependent on the storage of water in dams of
groundwater reservoirs, which are affected by long-term drought (e.g.,
SPI-36).

\hypertarget{drought-information-to-aid-in-adaptation}{%
\subsection{Drought information to aid in
adaptation}\label{drought-information-to-aid-in-adaptation}}

Different climate components, such as evaporative water demand, water
supply, soil moisture, and their impact on vegetation, should be
considered when evaluating the multi-dimensional nature of drought. For
a better understanding of the propagation of drought \citep{VanLoon2012}
from meteorological to hydrological drought, we should consider the
climatic response at different time scales, ranging from short to long.
Furthermore, we must make this information publicly available and update
it frequently to aid in the development of adaptation policies. The
drought observatory for agriculture and biodiversity of Chile (ODES)
shares this information aggregated for multiple administrative and
hydrological units across Chile
(\url{https://odes-chile.org/app/unidades}), with the goal of helping to
prepare for future climatic conditions.

\hypertarget{conclusion}{%
\section{Conclusion}\label{conclusion}}

There is a trend toward decreasing water supply (SPI, SPEI, and SSI) in
most of Chile; just in the southern part, there is an increase. The
trend is most strong in the north-central zone. The whole country showed
an increase in water demand (AED). The magnitude of the trend increases
over longer time scales, which is evidence that the deficit is impacting
the hydrological system in Chile. The trend in vegetation productivity
in the north-central area is affecting, to a higher degree, shrubland
and savanna, followed by croplands and forests. The most important
changes in land cover are the increase of forest in ``Sur'' and
shrubland and grassland in ``Centro;'' and the increase of savanna in
``Centro'' and ``Sur.''.

The drought explains about 30\% of the change in land cover type across
Chile for forest, grassland, shrubland, and savanna. There is no
evidence of an effect of drought on the change in cropland surface area.
The increase in AED is the main driver of the change in land cover,
followed by a reduction in precipitation and soil moisture. The drought
time scales vary regarding the land cover type.

The change in vegetation productivity has been severe in the
north-central part of the country for all land cover types, particularly
savanna, shrubland, and croplands. The anomaly in soil moisture over the
past 12 months is the main variable explaining these changes, followed
by anomalies in cumulated precipitation over one to two years. The
variation in AED seems to intensify the drought impact on vegetation
productivity.

The results of this study provide insightful information that would help
in developing adaptation measures for ecosystems in Chile to cope with
climate change and drought.

\hypertarget{acknowledgment}{%
\section{Acknowledgment}\label{acknowledgment}}

The National Research and Development Agency of Chile (ANID) funded this
study through the drought emergency project FSEQ210022, Fondecyt
Iniciación N°11190360, and Fondecyt Regular N°1210526.


\renewcommand\refname{References}
  \bibliography{references.bib}


\end{document}
