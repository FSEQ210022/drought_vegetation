% Options for packages loaded elsewhere
\PassOptionsToPackage{unicode}{hyperref}
\PassOptionsToPackage{hyphens}{url}
\PassOptionsToPackage{dvipsnames,svgnames,x11names}{xcolor}
%
\documentclass[
  sn-nature,
  numbered]{sn-jnl}

\usepackage{amsmath,amssymb}
\usepackage{iftex}
\ifPDFTeX
  \usepackage[T1]{fontenc}
  \usepackage[utf8]{inputenc}
  \usepackage{textcomp} % provide euro and other symbols
\else % if luatex or xetex
  \usepackage{unicode-math}
  \defaultfontfeatures{Scale=MatchLowercase}
  \defaultfontfeatures[\rmfamily]{Ligatures=TeX,Scale=1}
\fi
\usepackage{lmodern}
\ifPDFTeX\else  
    % xetex/luatex font selection
\fi
% Use upquote if available, for straight quotes in verbatim environments
\IfFileExists{upquote.sty}{\usepackage{upquote}}{}
\IfFileExists{microtype.sty}{% use microtype if available
  \usepackage[]{microtype}
  \UseMicrotypeSet[protrusion]{basicmath} % disable protrusion for tt fonts
}{}
\makeatletter
\@ifundefined{KOMAClassName}{% if non-KOMA class
  \IfFileExists{parskip.sty}{%
    \usepackage{parskip}
  }{% else
    \setlength{\parindent}{0pt}
    \setlength{\parskip}{6pt plus 2pt minus 1pt}}
}{% if KOMA class
  \KOMAoptions{parskip=half}}
\makeatother
\usepackage{xcolor}
\setlength{\emergencystretch}{3em} % prevent overfull lines
\setcounter{secnumdepth}{-\maxdimen} % remove section numbering
% Make \paragraph and \subparagraph free-standing
\makeatletter
\ifx\paragraph\undefined\else
  \let\oldparagraph\paragraph
  \renewcommand{\paragraph}{
    \@ifstar
      \xxxParagraphStar
      \xxxParagraphNoStar
  }
  \newcommand{\xxxParagraphStar}[1]{\oldparagraph*{#1}\mbox{}}
  \newcommand{\xxxParagraphNoStar}[1]{\oldparagraph{#1}\mbox{}}
\fi
\ifx\subparagraph\undefined\else
  \let\oldsubparagraph\subparagraph
  \renewcommand{\subparagraph}{
    \@ifstar
      \xxxSubParagraphStar
      \xxxSubParagraphNoStar
  }
  \newcommand{\xxxSubParagraphStar}[1]{\oldsubparagraph*{#1}\mbox{}}
  \newcommand{\xxxSubParagraphNoStar}[1]{\oldsubparagraph{#1}\mbox{}}
\fi
\makeatother


\providecommand{\tightlist}{%
  \setlength{\itemsep}{0pt}\setlength{\parskip}{0pt}}\usepackage{longtable,booktabs,array}
\usepackage{calc} % for calculating minipage widths
% Correct order of tables after \paragraph or \subparagraph
\usepackage{etoolbox}
\makeatletter
\patchcmd\longtable{\par}{\if@noskipsec\mbox{}\fi\par}{}{}
\makeatother
% Allow footnotes in longtable head/foot
\IfFileExists{footnotehyper.sty}{\usepackage{footnotehyper}}{\usepackage{footnote}}
\makesavenoteenv{longtable}
\usepackage{graphicx}
\makeatletter
\def\maxwidth{\ifdim\Gin@nat@width>\linewidth\linewidth\else\Gin@nat@width\fi}
\def\maxheight{\ifdim\Gin@nat@height>\textheight\textheight\else\Gin@nat@height\fi}
\makeatother
% Scale images if necessary, so that they will not overflow the page
% margins by default, and it is still possible to overwrite the defaults
% using explicit options in \includegraphics[width, height, ...]{}
\setkeys{Gin}{width=\maxwidth,height=\maxheight,keepaspectratio}
% Set default figure placement to htbp
\makeatletter
\def\fps@figure{htbp}
\makeatother

%%%% Standard Packages

\usepackage{graphicx}%
\usepackage{multirow}%
\usepackage{amsmath,amssymb,amsfonts}%
\usepackage{amsthm}%
\usepackage{mathrsfs}%
\usepackage[title]{appendix}%
\usepackage{xcolor}%
\usepackage{textcomp}%
\usepackage{manyfoot}%
\usepackage{booktabs}%
\usepackage{algorithm}%
\usepackage{algorithmicx}%
\usepackage{algpseudocode}%
\usepackage{listings}%

%%%%

\raggedbottom
\makeatletter
\@ifpackageloaded{caption}{}{\usepackage{caption}}
\AtBeginDocument{%
\ifdefined\contentsname
  \renewcommand*\contentsname{Table of contents}
\else
  \newcommand\contentsname{Table of contents}
\fi
\ifdefined\listfigurename
  \renewcommand*\listfigurename{List of Figures}
\else
  \newcommand\listfigurename{List of Figures}
\fi
\ifdefined\listtablename
  \renewcommand*\listtablename{List of Tables}
\else
  \newcommand\listtablename{List of Tables}
\fi
\ifdefined\figurename
  \renewcommand*\figurename{Figure}
\else
  \newcommand\figurename{Figure}
\fi
\ifdefined\tablename
  \renewcommand*\tablename{Table}
\else
  \newcommand\tablename{Table}
\fi
}
\@ifpackageloaded{float}{}{\usepackage{float}}
\floatstyle{ruled}
\@ifundefined{c@chapter}{\newfloat{codelisting}{h}{lop}}{\newfloat{codelisting}{h}{lop}[chapter]}
\floatname{codelisting}{Listing}
\newcommand*\listoflistings{\listof{codelisting}{List of Listings}}
\makeatother
\makeatletter
\makeatother
\makeatletter
\@ifpackageloaded{caption}{}{\usepackage{caption}}
\@ifpackageloaded{subcaption}{}{\usepackage{subcaption}}
\makeatother

\ifLuaTeX
  \usepackage{selnolig}  % disable illegal ligatures
\fi
\usepackage[]{natbib}
\bibliographystyle{plainnat}
\usepackage{bookmark}

\IfFileExists{xurl.sty}{\usepackage{xurl}}{} % add URL line breaks if available
\urlstyle{same} % disable monospaced font for URLs
\hypersetup{
  pdftitle={Shifts in water supply and demand drive land cover change across Chile},
  pdfauthor={Francisco Zambrano; Anton Vrieling; Francisco Meza; Iongel Duran-Llacer; Francisco Fernández; Alejandro Venegas-González; Nicolas Raab; Dylan Craven},
  pdfkeywords={drought, land cover, water demand, water supply},
  colorlinks=true,
  linkcolor={blue},
  filecolor={Maroon},
  citecolor={Blue},
  urlcolor={Blue},
  pdfcreator={LaTeX via pandoc}}


\title[Shifts in water supply and demand drive land cover change across
Chile]{Shifts in water supply and demand drive land cover change across
Chile}

% author setup
\author*[1,2]{\fnm{Francisco} \sur{Zambrano}}\email{francisco.zambrano@umayor.cl}\author[3]{\fnm{Anton} \sur{Vrieling}}\author[4,5,6]{\fnm{Francisco} \sur{Meza}}\author[7]{\fnm{Iongel} \sur{Duran-Llacer}}\author[8,9]{\fnm{Francisco} \sur{Fernández}}\author[10]{\fnm{Alejandro} \sur{Venegas-González}}\author[4]{\fnm{Nicolas} \sur{Raab}}\author[11,12]{\fnm{Dylan} \sur{Craven}}
% affil setup
\affil[1]{, \orgname{Hémera Centro de Observación de la Tierra, Facultad
de Ciencias, Escuela de Ingeniería en Medio Ambiente y Sustentabilidad,
Universidad Mayor}}
\affil[2]{, \orgname{Observatorio de Sequía para la Agricultura y la
Biodiversidad de Chile (ODES), Universidad Mayor}}
\affil[3]{, \orgname{Faculty of Geo-Information Science and Earth,
University of Twente}}
\affil[4]{, \orgname{Facultad de Agronomía y Sistemas Naturales,
Pontificia Universidad Católica de Chile.}}
\affil[5]{, \orgname{Instituto para el Desarrollo Sustentable.
Pontificia Universidad Católica de Chile}}
\affil[6]{, \orgname{Centro Interdisciplinario de Cambio Global,
Pontificia Universidad Católica de Chile}}
\affil[7]{, \orgname{Hémera Centro de Observación de la Tierra, Facultad
de Ciencias, Universidad Mayor,}}
\affil[8]{, \orgname{Center of Economics for Sustainable Development
(CEDES), Faculty of Economics and Government, Universidad San
Sebastian}}
\affil[9]{, \orgname{Center of Applied Ecology and Sustainability
(CAPES)}}
\affil[10]{, \orgname{Instituto de Ciencias Agroalimentarias, Animales y
Ambientales (ICA3), Universidad de O'Higgins}}
\affil[4]{}
\affil[11]{\orgdiv{GEMA Center for Genomics, Ecology \& Environment,
Universidad Mayor, Camino La Pirámide Huechuraba 5750}}
\affil[12]{, \orgname{Data Observatory Foundation}}

% abstract 

\abstract{Globally, droughts are becoming longer, more frequent, and
more severe, and their impacts are multidimensional. These impacts
typically extend beyond the water balance, as long-term, cumulative
changes in the water balance can lead to regime shifts in land cover.
Here, we assess the effects of temporal changes in water supply and
demand over multiple time scales on vegetation productivity and land
cover changes in continental Chile, which has experienced a severe
drought since 2010. Over most of continental Chile, we have observed a
persistent negative trend in water supply and a positive trend in
atmospheric water demand since 2000. However, in water-limited zones, we
have observed a negative trend in the vegetation's water demand. All the
trends intensify over longer time scales. This long-term decrease in
water availability has led to a decrease in vegetation productivity,
especially for the Chilean matorral and the Valdivian temperate forest.
Drought indices of soil moisture and actual evapotranspiration from
short- to mid-term time scales primarily explain this decrease. Besides,
our models indicate that the intensity of the drought explains up to
78\% of the changes in shrubland land covers, decreasing to 40\% for
forests, but the change in cropland surface is explained by the surface
burned. Long-term climate change may cause regime shifts in land cover,
which context-specific adaptation strategies can mitigate.}

% keywords
\keywords{drought,  land cover,  water demand,  water supply}

\begin{document}
\maketitle


\section{Introduction}\label{sec-intro}

Across many regions of the world, droughts are becoming longer, more
frequent, and more severe\citep{Miranda2023, IPCC2023}, impacting
ecosystems via tree mortality\citep{Cheng2024}, reducing vegetation
productivity\citep{Miranda2023} and inducing shifts in land use and
cover \citep{Crausbay2017}. However, identifying drought events is
idiosyncratic due to the varying criteria used for classification.
Droughts can be classified as 1) meteorological, i.e., when
precipitation in a specific period falls below mean precipitation values
observed over multiple years \citep{Mishra2010} (usually more than 30
years); 2) hydrological, i.e., when precipitation anomalies last for
long periods (months to years) and affect the hydrological system
\citep{VanLoon2015, VanLoon2016b} (e.g., streamflows, reservoirs and
groundwater); 3) agricultural, i.e.~when precipitation deficits
negatively impact plant health, leading to decreases in crop or pasture
productivity \citep{Wilhite1985}; or 4) ecological, i.e., when water
availability negatively affect the provisioning of ecosystem services
and trigger feedbacks in natural or human systems \citep{Crausbay2017}.
Such feedbacks include drought impacts on human decision making and
activities, which can lead to land-cover change
\citep{VanLoon2016, AghaKouchak2021}, which may have cascading effects
on biodiversity and ecosystem services
\citep[e.g.,][]{Lawler2014, Newbold2015}. Despite the high degree of
confidence in the impacts of rising temperatures on the extent,
frequency, and severity of agricultural and ecological droughts
\citep{IPCC2023}, which are likely to increase even if global warming
stabilizes at 1.5°--2°C, the severity of meteorological droughts has
been remarkably stable globally over the past century
\citep{Vicente-Serrano2022, Kogan2020}. A global study analyzing drought
severity trends from 1980 to 2020 reveals that in a few regions (some
mid-latitudinal and subtropical areas), rising temperatures during the
warm season have increased atmospheric evaporative demand (AED), leading
to an increase in agricultural land area \citep{Vicente-Serrano2022}.
Thus, rising atmospheric water demand may reflect parallel changes in
land cover---primarily agriculture---that can exacerbate the effects of
meteorological droughts on ecosystems.

Expanding analyses to include multiple dimensions of droughts can
provide complementary insights into the Earth's water balance - and its
impacts - over multiple time scales. Yet, the World Meteorological
Organization recommends the use of a single drought index for monitoring
droughts \citep{WMO2012}, i.e., the multi-scale Standardized
Precipitation Index (SPI; \citep{McKee1993}), which is limited in that
it only considers water supply in the form of precipitation. The
Standardized Precipitation Evapotranspiration Index (SPEI;
\citep{Vicente-Serrano2020}) builds upon SPI by incorporating the
effects of temperature on drought, and is now used widely for drought
monitoring \citep[e.g.,][]{Gebrechorkos2023, Liu2024}. Indices derived
from soil moisture products \citep{Narasimhan2005, Souza2021}, such as
the Standardized Soil Moisture Index
\citep[SSI,][]{AghaKouchak2014, AghaKouchak2015} also monitor water
supply and are thought to better capture water availability for crops,
thus providing more relevant information for evaluating agricultural
droughts. To disentangle the effects of precipitation from those of
temperature \citep{Vicente-Serrano2020}, as well as to capture droughts
in terms of water atmospheric demand, AED has been integrated into the
Evaporative Demand Drought Index \citep[EDDI,][]{McEvoy2016}, which is
particularly effective at detecting the rapid onset or intensification
of droughts. To quantify vegetation water demand, we must use the actual
evapotranspiration, or the amount of water removed from a surface by
evaporation and transpiration; the Standardized Evapotranspiration Index
(SETI; \citep{Yang2019}) can be used for this purpose. In turn,
ecological droughts, which capture the joint effects of precipitation
and temperature in modifying natural and productive ecosystems
\citep{Camps-Valls2021, Paruelo2016, Helman2014}, are complex to measure
and can therefore be monitored using multiple drought indices that
capture the multiple dimensions of drought, e.g., precipitation,
temperature, evapotranspiration, and AED. Although such an approach
accounts for the joint effects of changes in natural and productive
ecosystems, its potential impacts on land cover change have been largely
unexplored.

From 1960 to 2019, land-use change has impacted approximately one-third
of the Earth's surface, which is four times more than previously thought
\citep{Winkler2021}. Despite the considerable interest in land-use
change dynamics \citep[e.g.,][]{Song2018, Winkler2021}, the direction
and magnitude of drought impacts on land cover change and vegetation
productivity remain uncertain \citep{Chen2022, Akinyemi2021, Peng2017}.
Meteorological droughts are responsible for approximately 37\% of land
cover change and variability in vegetation productivity globally
\citep{Peng2017}. However, the evidence supporting these results is
derived from only one drought index, SPEI, which combines a proxy for
water supply - precipitation - with a proxy for water demand - AED - at
one time scale (12 months). The use of only one time scale may bias
results of drought impacts towards ecosystems dominated by plant growth
forms such as grasses and herbs that respond more rapidly to drought
stress (\textless{} 12 months). This is because physiological
differences among and within dominant plant growth forms may increase
(or decrease) tolerance of drought stress
\citep{Craine2013, McDowell2022}. For example, trees growing in more
arid ecosystems typically respond over longer time scales than those in
more humid ecosystems \citep{Vicente-Serrano2014}. Another source of
uncertainty regarding drought impacts on land cover change and
vegetation productivity are extrinsic factors, such as large-scale
public policy (e.g., national and international reforestation
initiatives), agricultural practices (e.g., clearing forest for soybean
or oil palm), and rural and urban land use planning
\citep{Karabulut2023}.

To deepen current knowledge on the multidimensional impacts of drought
on the temporal dynamics of natural and productive ecosystems, we
evaluate temporal changes in water supply and demand, net primary
productivity, and land-cover change across terrestrial ecosystems in
continental Chile for 2000-2023. Chile's diverse climate and ecosystems
\citep{Beck2023, Luebert2022} make it an ideal natural laboratory for
assessing the dynamic interactions between climate and ecosystems, and
potential impacts on land-cover change. Additionally, large parts of
Chile have experienced severe drought conditions that have significantly
affected vegetation and water storage in recent years; north-central
Chile has faced a persistent precipitation deficit (or ``mega-drought'')
since 2010 \citep{Garreaud2017}, which has broadly impacted native
forests \citep[e.g.,][]{Miranda2020, UrrutiaJalabert2018, Venegas2018}
and agricultural productivity
\citep[e.g.,][]{Zambrano2016, Zambrano2018, Zambrano2023}. However, the
effects of this prolonged extreme drought may also extend to changes in
land cover, altering the provision of key ecosystem services and
agricultural production. Here, we aim to assess: short- to long-term
time trends in multi-scalar drought indices that capture variation in
the components of water balance, i.e., water supply (SPI, SPEI, SSI) and
demand (EDDI, SETI) and their impacts on vegetation productivity and
land-cover change across continental Chile. By doing so, we aim to
deepen understanding of how water supply and demand affect vegetation
productivity and influence changes in the surface of land cover in a
severely drought-impacted region. This may enhance understanding of the
alterations, due to drought, to ecosystems in various regions across the
world.

\section{Materials and Methods}\label{materials-and-methods}

\subsection{Study area}\label{study-area}

Continental Chile has a diverse climate, with strong environmental
gradients from north to south and east to west \citep{Aceituno2021}
(Fig. 1a), which, together with its complex topography (Fig. 1b),
determine its ecosystem diversity \citep{Garreaud2009, Luebert2022}
(Fig. 1c). We therefore divided Chile into ecoregions
\citep{Dinerstein2017}, which are regions that share similar geography
and ecology, and have comparable levels of precipitation and solar
radiation. There are seven ecoregions: Atacama desert, Central Andean
dry puna, Southern Andean steppe, Chilean Matorral, Valdivian temperate
forests, Magellanic subpolar forests, and Patagonian steppe. The Atacama
desert is predominantly arid with hot (Bwh in the Koppen-Geiger
classification) and cold (Bwk) temperatures, as well as the northern
part of the Chilean Matorral. Most of the land in these two northern
regions is bare, except for a small area where shrublands and grasslands
are present. With an annual rainfall of less than 400 mm, the Central
Andean dry puna ecoregion has low, yet highly seasonal precipitation
with an eight-month dry season, low temperatures (Bwk) and is dominated
by grasslands, shrublands, and savanna. The climate of the Southern
Andean steppe ecoregion is cold desert (BWk), with most precipitation
occurring in the winter. There is little vegetation in this ecoregion
because the plants have adapted to its windy, dry, and cold climate. In
central Chile, the climate of the Chilean Matorral changes to that of an
arid steppe with cold temperatures (Bsk). Then, towards the center-south
of the country, the climate of the Chilean Matorral changes to a
Mediterranean climate, with warm to hot summers (Csa and Csb). Land
cover in this ecoregion consists of a significant amount of shrublands
and savannas. The Valdivian temperate forests have a mostly oceanic
climate (Cfb) and a large area of forests and grasslands. The Magellanic
subpolar forests have a tundra climate. Lastly, the Patagonian steppe
has high aridity, cold temperatures (Bsk), and primarily consists of
grasslands.

\subsection{Data}\label{data}

\subsubsection{Gridded meteorological and vegetation
data}\label{gridded-meteorological-and-vegetation-data}

To derive a proxy for vegetation productivity, we used the Normalized
Difference Vegetation Index (NDVI) from the MOD13A3 Collection 6.1
product derived from the MODIS (Moderate-Resolution Imaging
Spectroradiometer) sensor onboard the Terra satellite. MOD13A3 provides
vegetation indices with a 1 km spatial resolution and monthly frequency
\citep{Didan2015}. We also utilized the MOD16A2 collection 6.1
\citep{Running2017} product from MODIS to gauge the water consumption of
vegetation. This product gives us monthly actual evapotranspiration (ET)
and atmospheric evaporative demand (AED) with a \textasciitilde500m
spatial resolution. For soil moisture, water supply, and water demand
variables, we used ERA5-Land (ERA5L; ECMWF Reanalysis version 5 over
land) \citep{MunozSabater2021}, a reanalysis dataset that provides
atmospheric and land variables since 1950. It has a spatial resolution
of 0.1° (9 km), hourly frequency, and global coverage. We selected total
precipitation, maximum and minimum temperature at 2 meters, and
volumetric soil water layers between 0 and 100 cm of depth (layer 1 to
layer 3; Supplementary Materials and Methods, Supplementary Tables 2 and
4).

\subsubsection{Gridded indicators for land
use}\label{gridded-indicators-for-land-use}

To account for the impacts of human activity on land cover change, we
obtained data on road density \citep{Meijer2018} and night lights for
the period 2012--2023 \citep{Roman2018}. These products are frequently
used in the literature to quantify the human footprint (e.g.,
\citep{Halpern2015, Halpern2008}) or biodiversity threats (e.g.,
\citep{Bowler2020, Kennedy2020}).To capture changes on land cover due to
fires, we calculated the total burned area for 2002-2023
\citep{Chen2023}.

\subsection{Short- to long-term drought
trends}\label{short--to-long-term-drought-trends}

\subsubsection{Atmospheric Evaporative Demand
(AED)}\label{atmospheric-evaporative-demand-aed}

To compute drought indices that use water demand, it is necessary to
first calculate AED. To do this, we employed the Hargreaves method
\citep{Hargreaves1994, Hargreaves1985} by applying the following
equation:

\begin{equation}\phantomsection\label{eq-AED}{AED = 0.0023\cdot Ra\cdot (T+17.8)\cdot (T_{max}-T_{min})^{0.5}}\end{equation}

where \(Ra\) \((MJ\,m^2\, day^{-1})\) is extraterrestrial radiation;
\(T\), \(T_{max}\), and \(T_{min}\) are mean, maximum, and minimum
temperature \((°C)\) at 2m. For calculating \(Ra\) we used the
coordinate of the latitud of the centroid of each pixel as follow:

\begin{equation}\phantomsection\label{eq-Ra}{R_a = \frac{14,400}{\pi}\cdot G_{sc}\cdot d_r \left[\omega_s\cdot sin(\phi)\cdot sin(\delta)+cos(\phi)\cdot cos(\delta)\cdot sin(\omega_s) \right]}\end{equation}

where:

\(Ra\): extraterrestrial radiation \([MJ\, m^{-2} day-1]\),\\
\(G_{sc}\): solar constant = 0.0820 \([MJ\,m^{-2} min^{-1}]\),\\
\(d_r\): inverse relative distance Earth-Sun,\\
\(\omega_s\) sunset hour angle \([rad]\),\\
\(\phi\): latitude \([rad]\),\\
\(\delta\): solar declination \([rad]\).

We selected the method of Hargreaves to estimate AED because of its
simplicity, as it only requires temperature and extraterrestrial
radiation, and because access to the data needed for alternative methods
(e.g., Penman-Montieth) is often limited \citep{Vicente-Serrano2014}.

\subsubsection{Drought indices}\label{drought-indices}

To derive the drought indices of water supply and demand we used the
ERA5L dataset and the MODIS product (MOD13A3 \citep{Didan2015b}), with a
monthly frequency for 2000--2023. Drought indices capture historical
anomalies of water supply and demand. To quantify each anomaly, the
common practice is to derive it following a statistical parametric
method in which it is assumed that the statistical distribution of the
data is known \citep{Heim2002}. The use of an erroneous statistical
distribution that does not fit the data is usually the highest source of
uncertainty \citep{Laimighofer2022}. In the case of Chile, due to its
high degree of climatic variability, it is difficult to choose a
statistical distribution that can be used across its entire extent. We
therefore use a non-parametric method for the calculation of the drought
indices, following \citep{Farahmand2015}.

For monitoring water supply, we used the Standardized Precipitation
Index (SPI; \citep{McKee1993}), which only uses precipitation data. To
evaluate water demand, we chose the Evaporative Demand Drought Index
(EDDI; \citep{Hobbins2016} and \citep{McEvoy2016}), which is based on
AED, and the Standardized Evapotranspiration Index (SETI;
\citep{Yang2019}), which quantifies actual evapotranspiration, i.e.~the
amount of water removed from a surface due to evaporation and
transpiration. To quantify the combined effect of water supply and
demand, we estimated SPEI \citep{Vicente-Serrano2010}. For SPEI, we
calculated an auxiliary variable (D) with the following formula:

\[D = P-AED\] where P is precipitation. Soil moisture is often
considered to be the main driver of vegetation productivity,
particularly in semi-arid regions \citep{Li2022}. Hence, we used the
Standardized Soil Moisture Index (SSI) to estimate soil moisture (SM)
\citep{Hao2013}. For SSI, we used the average soil moisture from ERA5L
at a depth of 1m. All calculated indices are multi-scalar and can be
used for the analysis of short- to long-term droughts.

To derive the drought indices, we first calculate the sum of the
variables with regard to the time scale(s). In this case, for
generalization purposes, we will use \(V\), referring to variables P,
AED, D, and SM (Table SSX). We accumulated each over the time series of
values (months), and for the time scales s:

\begin{equation}\phantomsection\label{eq-sumvar}{A^s_i = \sum_{i=n-s-i+2}^{n-i+1} V_i\,\, \forall\, i\geq n-s+1  }\end{equation}

The \(A^s_i\) corresponds to a moving window (convolution) that sums the
variable for time scales \(s\). This is summed over \(s\) months,
starting from the most recent month (\(n\)) back in time until month
\(n-s+1\). For example, using as a variable the precipitation, a period
of twelve months (\(n\)), and a time scale of three months (\(s\)):

\[
\begin{split}
A^3_1 &= P_{oct} +P_{nov} +P_{dic} \\
\vdots\,\,\, &= \,\,\,\vdots + \,\,\,\vdots + \,\,\,\vdots \\
A^3_{10} &= P_{jan}+P_{feb} +P_{mar}
\end{split}
\]

Then, we used the empirical Tukey plotting position \citep{Wilks2011}
over \(A_i^s\) to derive the \(P(A_i^s)\) probabilities across a period
of interest:

\begin{equation}\phantomsection\label{eq-probPai}{P(A^s_i) = \frac{i-0.33}{n+0.33'}}\end{equation}

An inverse normal approximation \citep{Abramowitz1968} obtains the
empirically derived probabilities once the variable cumulates over time
for the scale \(s\). Thus, the drought indices \(SPI\), \(SPEI\),
\(EDDI\), and \(SSI\) are obtained following the equation:

\begin{equation}\phantomsection\label{eq-DI}{DI(A^s_i) = W - \frac{C_0+C_1\cdot W + c_2 \cdot W^2}{1+d_1\cdot W +d_2\cdot W^2 +d_3\cdot W^3}}\end{equation}

\(DI\) is referring to the drought index calculated for the variable
\(V\). The values for the constats are: \(C_0 = 2.515517\),
\(C_1 = 0.802853\), \(C_2 = 0.010328\), \(d_1 = 1.432788\),
\(d_2 = 0.189269\), and \(d3 = 0.001308\). For \(P(A^s_i) \leq 0.5\),
W=\(\sqrt{-2\cdot ln(P(A^s_i))}\) , and for \(P(A^s_i) > 0.5\), replace
\(P(A^s_i)\) with \(1-P(A^s_i)\) and reverse the sign of \(DI(A^s_i)\).

The drought indices were calculated for time scales of 1, 3, 6, 12, 24,
and 36 months at a monthly frequency for 2000--2023.

\subsubsection{Temporal trends of drought
indices}\label{temporal-trends-of-drought-indices}

To determine if there are statistically significant positive or negative
temporal trends for the drought indices, we used the non-parametric
modified Mann-Kendall test for serially correlated data \citep{Yue2004}.
To determine the magnitude of the trend, we used Sen's slope
\citep{Sen1968}. Sen's slope is less affected by outliers than
parametric ordinary least squares (OLS) regression, and as a
non-parametric method, it is not influenced by the distribution of the
data. We applied both methods for SPI, EDDI, SPEI, SETI, and SSI and six
time scales, resulting in a total of 30 trends. We then aggregated
temporal trends for each ecoregion and land cover type.

\subsubsection{Vegetation productivity}\label{vegetation-productivity}

We also used the MODIS product to calculate vegetation productivity, and
calculated anomalies of cumulative NDVI using zcNDVI
\citep{Zambrano2018}, which was derived from the monthly time series of
NDVI, with Equations 2 and 4. For vegetation productivity, we selected
the time scale that best correlates with annual net primary productivity
(NPP) across continental Chile. For this purpose, we calculated zcNDVI
for time scales of 1, 3, 6, and 12 months (from December) and compared
it with the annual NPP. We obtained NPP from MOD17A3HGF90. We chose to
use six months because the \(R^2\) between zcNDVI and NPP reaches its
highest value at six months, obtaining an \(R^2\) of 0.31 for forest and
0.72 for shrubland (Supplementary Information Section S2). We
subsequently used zcNDVI with a time scale of 6 months and calculated it
at a monthly frequency for 2000--2023.

\subsection{Drought impacts on vegetation
productivity}\label{drought-impacts-on-vegetation-productivity}

For each land cover, we analyzed the trend of vegetation productivity.
To this end, we identified areas within each land cover that are
persistent over time to reduce the possibility that trends in vegetation
productivity may be influenced by changes in land cover. We examined the
correlation between drought indices and vegetation productivity across
land cover types to determine to the extent to which soil moisture and
water demand and supply affect vegetation productivity.

We estimated pixel-to-pixel Pearson's correlations between drought
indices at time scales of 1, 3, 6, 12, 24, and 36 months with zcNDVI. We
extracted the Pearson correlation coefficient corresponding to the time
scale with the highest value. For each index, we then generated two
maps: 1) a raster with values of the time scales and drought index that
reached the maximum correlation, and 2) a raster with the magnitude of
the correlation between the drought index and vegetation productivity.

\subsection{Drought impacts on land cover
change}\label{drought-impacts-on-land-cover-change}

\subsubsection{Land cover change}\label{land-cover-change}

To analyze land cover change, we used the classification scheme of the
International Geosphere-Biosphere Programme (IGBP) from the product
MCD12Q1 Collection 6.1 from MODIS. The MCD12Q1 product is produced for
each year from 2001 to 2023 and defines 17 classes (see Table S1).
Following the FAO classification \citep{FAO2022}, we classified native
and planted forests as ``forests'', which represent natural and
productive ecosystems dominated by large trees. To analyze the land
cover change, we use the IGBP scheme from the MCD12Q1 product. We
regrouped the 17 classes into ten macro-classes, as follows: 1-4 to
forests (native forest and plantations), 5-7 to shrublands, 8-9 to
savannas, 10 as grasslands, 11 as wetlands, 12 and 14 to croplands, 13
as urban, 15 as snow and ice, 16 as barren, and 17 as water (Table S3).
This resulted in a time series of land cover with ten macro-classes for
2001-2023. We validated the land cover macro-classes using a high
resolution (30 m) land cover map for 2013-2014 \citep{Zhao2016}. Our
results showed a global accuracy of \textasciitilde0.82 and a F1 score
of \textasciitilde0.66 (Supplementary Information, S2).

We calculated the area for each land cover class in the five study
regions for 2001--2023. We then estimated the temporal change in area
for each land cover type and macro-class, and determined the statistical
significance (p-value \textless{} 0.05) and magnitude of the trend as
described above.

To assess how water demand and supply, and soil moisture affect the
variation in vegetation productivity across various land cover types, we
avoid analyzing areas that experienced major land cover changes in the
2001--2023 period. To assess how zcNDVI varied irrespective of land
cover change, we developed a persistence mask for land cover, which only
retains pixels for which the macro-class remained the same for at least
80\% of the 23 years (Fig. 1d).

\subsection{Relationship between land cover and drought
trends}\label{relationship-between-land-cover-and-drought-trends}

To identify which drought indices and time scales have a major impact on
changes in land cover type, we examined the relationships between the
temporal trends in the surface of land cover classes, drought indices,
road density, burned area, and night lights, and for each ecoregion. We
performed the analysis at the sub-basin scale, using 485 river basins,
which have a surface area between 0.906 and 24,408 km2 and a median area
of 1,249 km2 (Supplementary Fig. S8/Table S5). For each basin, we
calculated the trend per land cover, considering the proportion of the
type relative to the total surface of the basin. For each basin we
extracted the average trend of all drought indices and at time scales of
1, 3, 6, 12, 24, and 36 months. In the case of burned area, we used as
variables the total and the trend of burned area for 2002-2023, and for
night lights we used the average and the trend of nightlights for
2012-2023.

Prior to modelling relationships between trends in land cover and
drought indices, we assessed multi-collinearity among explanatory
variables, i.e., drought indices, road density, night lights, and burned
area, with the variance inflation factor (VIF). We analyzed the VIF for
all drought indices at time scales of 1, 3, 6, 12, 24, and 36 separately
because each index has a strong correlation across time scales. As VIF
values greater than five may affect the interpretation of model results
\citep{Dorman2013}, we therefore excluded SPI from all subsequent
models.

To assess the relationship between land cover trends and drought
indices, we modeled trends in the surface of land cover types. We made a
regression analysis using the random forest method93, which employs
multiple decision trees. Some advantages of random forest include the
ability to find non-linear relationships, reduce overfitting, and derive
variable importance. We incorporated the trends of the five drought
indices (SPI, SPEI, EDDI, SETI, and SSI), the trends of night lights (2)
and burned area (2), the road density, for a total of ten predictors. We
then constructed random forest models for each time scale (1, 3, 6, 12,
24, and 36) and each land cover class (forest, grassland, shrubland,
savanna, cropland, and barren land), resulting in a total of 36 RF
models. We trained each model using 1000 trees, setting the minimum
number of nodes per decision tree at five and the number of predictors
per split (boosting) to the square root of the total number of
predictors. To account for uncertainty, we trained all the models ten
times using a resampling strategy (ten folds) in a cross-validation
scheme. Finally, we evaluate model fit by calculating R², root mean
square error (RMSE), and variable importance. Variable importance
identifies which variables have a higher contribution to explaining
model variation. We calculated variable importance by permuting
out-of-bag (OOB) data per tree and calculating the mean standard error
of the OOB data. After permuting each predictor variable, we repeated
the process for the remaining variables. We repeated this process ten
times per fold to assess model fit.

Finally, we visually explored the relationship between drought indexes
and changes in land cover across sub basins within Chile. To achieve
this, we compared the relative changes in land cover surface with the
drought indices, burned area, nighlights, and road density and other
variables for the time scale that were deemed more significant in the
random forest model.

\subsection{Software}\label{software}

For downloading, processing, and analysis of the spatio-temporal data,
we used the open source software for statistical computing and graphics,
R \citep{R2023}. For downloading ERA5L, we used the \{ecmwfr\} package
\citep{Hufkens2019}. For processing raster data, we used \{terra\}
\citep{Hijmans2023} and \{stars\} \citep{Pebesma2023}. For managing
vectorial data, we used \{sf\} \citep{Pebesma2018}. For the calculation
of AED, we used \{SPEI\} \citep{Bergueria2023}. For mapping, we used
\{tmap\} \citep{Tennekes2018}. For data analysis and visualization, the
suite \{tidyverse\} \citep{Wickham2019} was used. For the random forest
modeling, we used the \{tidymodels\} \citep{Kuhn2020} and \{ranger\}
\citep{Wright2017} packages.

\section{Results}\label{results}

\subsection{The Chilean matorral and Patagonian steppe increase
atmospheric water demand but decrease vegetation
evapotranspiration}\label{the-chilean-matorral-and-patagonian-steppe-increase-atmospheric-water-demand-but-decrease-vegetation-evapotranspiration}

For the Atacama desert and the Central Andean Puna, we found a positive
temporal trend for drought indices of water supply (i.e., SPI, SSI),
atmospheric water demand (i.e., EDDI), and vegetation water demand
(i.e., SETI). For the Chilean matorral and Patagonian steppe, EDDI
presents the higher positive trend and lower negative trends in SPI,
SPEI, SSI, and SETI. This reflects a critical scenario of drought, where
a rise in temperature increases atmospheric water demand, but vegetation
cannot increase evapotranspiration due to a lack of water availability.
In the Southern Andean steppe, there is a positive trend in AED (i.e.,
EDDI), but a negative in water supply (i.e., SPI, SPEI, SSI). The
vegetation water demand (i.e., SETI) has negative trends, but it is
increasing at higher time scales. The Valdivian temperate forests show a
negative trend in water supply (i.e., SPI, SPEI, and SSI) and a positive
trend in both AED and ET, as shown by the EDDI and SETI, respectively.
Here, an increase in AED implies an increase in ET, likely due to a
greater availability of water, unlike in the Chilean Matorral and
Patagonian steppe. The vegetation water demand (SETI) in the Magellanic
subpolar forests does not exhibit a significant trend over any given
time scale. The AED and water supply present a positive trend. The
trends of drought indices in the Patagonian steppe exhibit a similar
behavior to the Chilean matorral, albeit to a lesser extent. Generally,
the majority of the indices indicate that the trend (positive or
negative) intensifies over longer time periods. (Fig. 2)

\subsection{Vegetation productivity has strongly decreased in the
Chilean matorral and the Patagonian
steppe.}\label{vegetation-productivity-has-strongly-decreased-in-the-chilean-matorral-and-the-patagonian-steppe.}

We found contrasting temporal trends in vegetation productivity for
2000-2023 (Fig. 3). The Atacama desert does not exhibit significant
trends over time. The Chilean Matorral, Patagonian steppe, and the
Southern Andean steppe exhibit negative trends of -0.023, -0.016, and
-0.006 (z-score/per decade), respectively. In contrast, the Central
Andean dry puna, Valdivian temperate forests, and Central Andean dry
puna show positive trends ranging from 0.01 to 0.03 (z-score/per
decade). The Chilean matorral was at its lowest point from 2019 to 2022,
while the Patagonian steppe has experienced an increasingly negative
trend since 2022.

Forest, savanna, and shrubland exhibit the highest change in surface
area across ecoregions We also observed significant changes in land
cover surfaces across continental Chile (Fig. 4). The forest surface has
increased in the Chilean matorral and in the Valdivian temperate forest
at rates of 78 and 316 km² yr⁻¹, respectively. Grassland surface has
diminished in the Southern Andean steppe (-19 km² yr⁻¹) and has
increased in the Patagonian steppe (90 km² yr⁻¹). Savanna exhibits a
decrease in the Chilean matorral of -271 km² yr⁻¹ and the Valdivian
temperate forest of -276 km² yr⁻¹, but an increase at a rate of 133 km²
yr⁻¹ in the Magellanic subpolar forest. Shrubland has the highest
increase in the Chilean matorral (160 km² yr⁻¹). Barren land has
increased in the Central Andean dry puna (36 km² yr⁻¹) and the Southern
Andean steppe (50 km² yr⁻¹), but has diminished in the Magellanic
subpolar forest (-81 km² yr⁻¹). \#\# Drought impact on vegetation
productivity are strongest in the Chilean matorral and Valdivian
temperate forest

Our results indicate that drought impacts on vegetation productivity are
highest in the Chilean Matorral and Valdivian temperate forests across
all land cover types, except forest (Fig. 5, Fig. S5 and Table 1). For
time scales of 6 and 12 months, SETI and SSI have the strongest positive
correlation with vegetation productivity among the land cover types.
Next, we found that grassland and savanna in the Patagonian steppe had
higher correlations with SPI and SSI over 12 months. Further, there is a
positive relationship between the vegetation in the Atacama desert and
drought indices of 12 months of water supply and vegetation water
demand. However, there is a negative relationship between the vegetation
and atmospheric water demand over 12 months. All drought indices show a
positive correlation with the vegetation in the Central Andean dry puna,
particularly the drought indices of water supply (SPI, SPEI, and SSI) at
time scales of 24 and vegetation water demand (SETI) at time scales of
36 months. For the Southern Andean steppe the SETI of 24 months showed
the highest correlation with savannas, followed by the EDDI of 24
months. Our analysis also revealed that water demand and supply
differentially affected the time scales at which vegetation productivity
of land cover types within each region was most impacted by drought
(Fig. 5, Fig. S5 and Table 1). While the spatial variation in the
relationship between drought intensity and vegetation productivity was
consistent across drought indices, the drought indices that captures
water supply via soil moisture (Standardized Soil Moisture Index; SSI),
and via vegetation water demand (Standardized Evapotranspiration Index,
SETI) tended to show a stronger correlation with vegetation productivity
over larger areas than the other drought indices (Fig. 5).

\subsection{Drought strongly impacts land cover distribution for
shrubland}\label{drought-strongly-impacts-land-cover-distribution-for-shrubland}

Our random forest models show that drought indices explain between
32-79\% of the variation in land cover change across continental Chile
(Fig. 6). Moreover, these results highlight the importance of
considering water supply and demand, as drought indices associated with
both aspects of the water balance had high importance values across most
ecoregions and land cover types. The variation in the time scale of
drought indices that provide the strongest correlation with vegetation
productivity also suggests that different types of vegetation are not
equally sensitive to droughts of similar intensities (Table 1). The RF
models show that the drought indices are explaining 71-78\% of the
variability in land cover surface change for shrublands. Second, the RF
models explain approximately 58-78\% of the variability in change on
croplands. In the case of other land cover types, the RF models account
for approximately 33-59\% of the variability, with drought indices
explaining less for the forest type (Fig. 6).

Our model has the highest r-squared for shrublands, followed by
croplands, and barren land (Fig. 6). Ecoregions most frequently observed
the variables SETI and SSI, which significantly influenced surface
changes for each land cover (Fig. 7). In cropland, the primary factor is
not the drought indices, but rather the total surface or trend of the
burned area. The nightlight variable, which reflects urban areas,
primarily explains the change in the surface of barren land, followed by
the drought indices SPEI at time scales of 3 and 6 months (Fig. 7).

Our results also show that drought intensity was associated with the
magnitude and direction of land cover change (Fig. 8). We observe that
shrublands are sensitive to both increases and decreases in SETI and
SSI, reaching a point of equilibrium around a normal climatic situation
(drought index = 0). This could potentially stem from the fact that
favorable water supply conditions alter the type and quantity of
vegetation. Conversely, a reduction in water supply results in the
shrublands losing their vegetation and transforming into bare soil.
Thereby, both situations alter the land cover type. Changes in the
burned surface impact the cropland surface; an increase in the burned
area leads to an increase in the agricultural surface. Agricultural
areas likely replace the burned area, explaining this phenomenon. In the
case of bare soil, values that indicate low urban development
(nightlights) are associated with a larger surface of bare soil.
However, when the variable's range extends beyond the normal urban
situation (zero value), an increase in bare soil correlates with a
fluctuating increase in urban development. The relationship between SETI
and SPEI in grasslands is opposite; an increase in SPEI leads to an
increase in grassland area, while an increase in water demand by
grasslands, as reflected by SETI, results in a decrease in grassland
area.

\section{Discussion}\label{discussion}

\subsection{Temporal trends in water supply and
demand}\label{temporal-trends-in-water-supply-and-demand}

We discovered that the Atacama desert, Central Andean dry puna, and the
Magellanic subpolar forests experience an increase in water supply (SPI,
SSI), as well as an increase in atmospheric and vegetation water demand
(EDDI, SETI). However, in the Magellanic subpolar forests, there is no
significant increase or decrease in SETI across time scales. Also, we
found a significant decreasing trend in water supply (SPI, SPEI, and
SSI) across the Southern Andean steppe, Chilean Matorral
\citep{Boisier2018, Sarricolea2019}, Valdivian temperate forests, and
Patagonian steppe, accompanied by an increase in atmospheric water
demand (EDDI). Our results indicate that the trend of water supply and
atmospheric demand tends to decrease or increase more strongly over
longer time scales---a trend that is consistent with the progressive
intensification of drought severity across much of Chile, as observed in
other regions facing long-term droughts \citep{Rashid2019, Miro2023}.
Simultaneously, we observed a divergent trend between the EDDI and SETI
in various ecoregions. In the majority of ecoregions, a rise in
atmospheric water demand (EDDI) typically leads to a rise in vegetation
water demand (SETI). However, in the zones most affected by drought
(Fig. 3 and Fig. 5), where water availability is limited, an increase in
atmospheric water demand results in a decrease in vegetation water
demand. The Chilean matorral and the Patagonian steppe are where this is
most evident. When combined, our findings demonstrate a persistent
drying trend in the Chilean matorral, the Patagonian steppe, and the
Southern Andean steppe. We attribute this trend to a simultaneous
decrease in precipitation and an increase in atmospheric evaporative
demand, which leads to a decrease in the water demand for vegetation in
water-limited zones\citep{Pascoa2021}.

\subsection{Temporal trends in vegetation
productivity}\label{temporal-trends-in-vegetation-productivity}

The consequences of the persistent drying trend for ecosystems
throughout continental Chile are manifold. First, the prolonged
hydrological drought, i.e., precipitation deficit, has reduced
groundwater storage (SSI, \citep{Taucare2024}), leading to a steady
decline in vegetation productivity (zcNDVI) since 2000 across the
Chilean Matorral, Patagonian steppe, and Southern Andean steppe,
reaching its lowest level between 2020 and 2022, which could be due to
either a decrease in vegetation area, a loss of biomass or browning in
forest ecosystems. Reports for natural and productive ecosystems
\citep{Nicolai-Shaw2017, Jiang2020, Zhou2021} most strongly associate
this decline with declines in soil moisture and increases in
evapotranspiration. Second, the sharp decline in vegetation productivity
in the Chilean Matorral and Valdivian temperate forests showed that
grasses and shrublands respond to shifts in water supply on higher time
scales (12 months), in comparison to savanna and croplands (6 months).
Also, in the Valdivian temperate forest that has a higher surface with
forest trees, soil moisture (SSI) and vegetation water demand (SETI) at
higher time scales, 12 and 36 months, respectively, evidence stronger
correlations. Which is consistent with recent studies showing that
progressive, long-term water deficits in central Chile have triggered
forest browning and declines in native forest productivity
\citep[e.g.,][]{Miranda2023, Miranda2020, Venegas2022}. While our
analysis do not distinguish between native and planted forests, the
latter of which are considered to be more drought tolerant in central
and southern Chile \citep{Carrasco2022}, we show that forest area
declines more sharply in response to increasing water demand due to
rising temperatures (EDDI) than decreasing water supply (e.g., SPI, SSI;
refs. \citep{Fajardo2019, Holz2018}), which may have cascading impacts
on multiple facets of forest diversity \citep{Segovia2021, Sabatini2022}

Moreover, the strengthening of the correlation between vegetation
productivity and water supply (SPI, SPEI, SSI) or demand (EDDI, SETI)
over multiple time scales (up to 36 months) and across land cover types
(Fig. 5) - demonstrates the impacts of climate change on the water
balance in Chile. Impacts may extend beyond vegetation productivity, as
reduced soil moisture in central Chile and the western United States has
increased wildfire activity\citep{Holden2018, Gonzalez2018}, which is a
growing concern in Chile and may be further exacerbated by extensive
plantations of highly flammable tree species, e.g., Eucalyptus spp. and
Pinus spp. \citep{Bowman2019}. Third, we found that the decline in the
vegetation productivity of croplands is due to a decrease in the water
supply and vegetation water demand to a greater extent than to an
increase in atmospheric water demand \citep{Quiring2010}, causing a
decline in water availability. This is consistent with evidence that
more water-intensive crops have replaced less water-intensive crops in
central Chile, leading to an increase in water extraction from rivers or
groundwater \citep{Munoz2020, Duran2020}.

\subsection{Drought impacts on land
cover}\label{drought-impacts-on-land-cover}

We found evidence that temporal decreases in water supply (SPEI, SSI)
and decreases in vegetation water demand (SETI) are driving shifts not
only in vegetation productivity but also in land cover across most of
continental Chile. Shrubland covers were the land cover type most
sensitive to changes in the water balance over short and long temporal
scales. In contrast, forest surfaces are the type less affected by
drought. Moreover, our results provide evidence that, in addition to
shrublands cover, other land cover types have been affected by water
deficits, particularly savannas and grasslands. Our results therefore
suggest that multiple land cover types could be vulnerable to regime
shifts towards more drought tolerant land cover types
\citep{Scheffer2001, Martinez-Vilalta2016}, such as forest, whose cover
increased non-linearly in response to increasing drought intensity.
Changes in cropland cover are not a direct consequence of drought (Fig.
7), but rather an indirect one, likely occupying burned areas and
possibly reflecting the decision of resource-poor farmers to migrate to
regions with more abundant water resources or to change economic
activity \citep{AghaKouchak2021, Hermans2021}. In contrast, the increase
in shrubland cover due to a decrease in savanna cover may be ecological,
as shrubs may be more drought tolerant than other growth
forms\citep{Eldridge2011, Gotmark2016}.

We run each model ten times to account for the uncertainty in its
performance. In the supplementary material (Fig. S12-S17), we show the
error bar of r-squared; despite the inherent uncertainty, the analysis
reinforces our results regarding how drought impacts land cover type.

Overall, our results show that long-term declines in water supply and
demand have induced widespread, multi-dimensional impacts on the
vegetation productivity and on the extent of land cover types. While
prolonged droughts may directly cause shifts to more drought-tolerant
land cover types, such as shrublands, they may also influence land cover
change through human decision making and activities. This study extends
current understanding of drought impacts by demonstrating how their
multidimensionality emerges over multiple time scales and across land
cover types, which can contribute to developing context-specific
adaptation strategies for agriculture, biodiversity conservation, and
natural resource management.

\subsection{Study limitations}\label{study-limitations}

This study has evaluated the impact of water supply and demand on
vegetation productivity and how land use change influences it, with some
important limitations to highlight. One of the major limitations of the
study is the consideration of secondary information. For instance, we
used estimates of water supply and demand, such as ERA5L and MODIS,
which, despite their improved estimation capacity, suffer from biases
and uncertainties \citep{Gomis2023, Clelland2024} in different areas or
climatic conditions. In this study, we compared the ERA5L data with
climatic stations (see Table S2) to verify bias and uncertainty, but
future studies will need to correct the products for a more precise
measurement. We used zcNDVI \citep{Zambrano2018} (MODIS) as a proxy for
vegetation productivity, which has proven to be a good estimate of NPP
(see Fig. S1 and S2), but its quality varies between different types of
vegetation.

To define land cover, products were used that estimate the classes using
classification models, which are subject to quality errors that must be
taken into account \citep{Stehman2019, Verburg2011}. In addition, in our
case we used macro classes of land cover, where, for example, the
different types of forests were grouped (e.g., monoculture, native
forest). This approach hinders our ability to understand the effects of
drought on the various subclasses within each land use class. When it
comes to cultivated areas, we cannot distinguish between rainfed and
irrigated areas using macro classes. However, in this study, we aimed to
provide a broad overview at a national scale, with the goal of focusing
on a more detailed level in future research.

Other studies have analyzed how human decisions impact land use change,
such as population increase \citep{Kleemann2017}. In our case, we
attempted to consider this by incorporating road density, night lights,
and burned areas. However, human decisions are complex to consider from
the point of view of earth observation. On the other hand, earth
observation tools can analyze land cover change, but each type's use
depends on social and economic factors, which are challenging to
quantify \citep{Rindfuss2004, Jenner2024} and necessitate the
integration of social, natural, and geographic information sciences.

\section{Data availability}\label{data-availability}

The codes generated during the current study are available in the GitHub
repository, https://github.com/FSEQ210022/drought\_vegetation. The
datasets generated and/or analyzed during the current study are
available in the Zenodo repository,
https://doi.org/10.5281/zenodo.10359547.

\section*{Acknowledgments}\label{acknowledgments}
\addcontentsline{toc}{section}{Acknowledgments}

The National Research and Development Agency of Chile (ANID) funded this
study through the drought emergency project FSEQ210022, Fondecyt
Iniciación N°11190360, Fondecyt Postdoctorado N°3230678, and Fondecyt
Regular N°1210526.


  \bibliography{referencias.bib}



\end{document}
